\PassOptionsToPackage{paperwidth=1080pt,%
                      paperheight=1920pt,%
                      margin=0pt}{geometry} %   <-- 1
\PassOptionsToPackage{landscape}{geometry}

\documentclass{mybeamer}

\usepackage{mybeamer}

\begin{document}

\begin{frame}
  \centering
  {\Huge C\# Flashcards 2025 edition}\\[0.5em]
  {\large by \href{https://github.com/konradcinkusz}{\faGithub\;dev\_insight}}
\end{frame}

\begin{frame}{About This Deck}
  \setbeamercolor{block title}{bg=cyan!80!black,fg=white}
  \setbeamercolor{block body}{bg=cyan!10,fg=black}

  \begin{block}{\faBullseye\quad Purpose}
    Learn and teach C\# through bite-sized flashcards.
  \end{block}

  \begin{block}{\faPlay\quad How to Use}
    Read the question prompt, then advance the slide to reveal the answer.
  \end{block}

  \begin{block}{\faList\quad Coverage}
    C\# language, threading, LINQ, EF, design principles \& patterns.
  \end{block}

   {
       \setbeamercolor{block title}{bg=green!70!black,fg=white}%
       \setbeamercolor{block body}{bg=green!10,fg=black}%
       \begin{block}{\faUsers\quad Contribute}
         \begin{itemize}
           \item \textbf{Code:} \href{https://github.com/konradcinkusz/CSharp-FlashCards-2025}{\color{blue!80!black}\faGithub\;\underline{CSharp-FlashCards-2025}}
           \item \textbf{Collaborate:} Fork the repo, open issues, or submit pull requests!
         \end{itemize}
       \end{block}
   }
\end{frame}

\begin{frame}[label=toc]{Table of Contents}
  \centering\setlength\tabcolsep{1em}
  \begin{tabular}{cc}
    \TOCButtonTall{sec1}{sec1}{C\# Beginner} &
    \TOCButtonTall{sec2}{sec2}{C\# Intermediate} \\[1em]
    \TOCButtonTall{sec3}{sec3}{C\# Advanced} &
    \TOCButtonTall{sec4}{sec4}{LINQ} \\[1em]
    \TOCButtonTall{sec5}{sec5}{Threading \& Async/Await} &
    \TOCButtonTall{sec6}{sec6}{Entity Framework} \\[1em]
    \TOCButtonTall{sec7}{sec7}{Design Principles} &
    \TOCButtonTall{sec8}{sec8}{Design Patterns} \\[1em]
    \TOCButtonTall{sec9}{sec9}{OAuth} &
    \TOCButtonTall{sec10}{sec10}{Microservices}
  \end{tabular}
\end{frame}

\hypertarget{sec1}{}
\section{C\# Language: Beginner}
% 1 ── What is C#?
\QuestionSlide[\CategoryBadge[OOPColor!20]{Language \& Platform}]
  {* What is C\#?}
\begin{frame}[fragile]
  \frametitle{%
    \begin{tikzpicture}[remember picture,overlay]
      \node[anchor=north east,xshift=-0.4cm,yshift=-0.4cm] at (current page.north east) {
        \CategoryBadge[OOPColor!20]{Language \& Platform}
      };
    \end{tikzpicture}

    Answer \theqcounter: * What is C\#?%
  }

  {\footnotesize
    C\# is a modern, object-oriented programming language developed by Microsoft as part of the .NET platform. It supports strong static typing, garbage collection, LINQ for data queries, async/await for asynchronous programming, and deep integration with the .NET ecosystem.
  }
\end{frame}

% 2 ── What is the .NET Framework?
\QuestionSlide[\CategoryBadge[OOPColor!20]{Language \& Platform}]
  {* What is the .NET Framework?}
\begin{frame}[fragile]
  \frametitle{%
    \begin{tikzpicture}[remember picture,overlay]
      \node[anchor=north east,xshift=-0.4cm,yshift=-0.4cm] at (current page.north east) {
        \CategoryBadge[OOPColor!20]{Language \& Platform}
      };
    \end{tikzpicture}

    Answer \theqcounter: * What is the .NET Framework?%
  }

  {\footnotesize
    The .NET Framework is Microsoft’s original managed runtime for Windows applications. It includes the Common Language Runtime (CLR), a comprehensive base class library, and development tools for building desktop and server applications.
  }
\end{frame}

% 3 ── What is the CLR?
\QuestionSlide[\CategoryBadge[OOPColor!20]{Language \& Platform}]
  {* What is the Common Language Runtime (CLR)?}
\begin{frame}[fragile]
  \frametitle{%
    \begin{tikzpicture}[remember picture,overlay]
      \node[anchor=north east,xshift=-0.4cm,yshift=-0.4cm] at (current page.north east) {
        \CategoryBadge[OOPColor!20]{Language \& Platform}
      };
    \end{tikzpicture}

    Answer \theqcounter: * What is the Common Language Runtime (CLR)?%
  }

  {\footnotesize
    The CLR is the execution engine for .NET applications. It loads and runs IL code, performing JIT compilation, garbage collection, security checks, exception handling, and interoperability with unmanaged code.
  }
\end{frame}

% 4 ── What is the CTS?
\QuestionSlide[\CategoryBadge[MemoryColor!20]{Types \& Memory}]
  {* What is the Common Type System (CTS)?}
\begin{frame}[fragile]
  \frametitle{%
    \begin{tikzpicture}[remember picture,overlay]
      \node[anchor=north east,xshift=-0.4cm,yshift=-0.4cm] at (current page.north east) {
        \CategoryBadge[MemoryColor!20]{Types \& Memory}
      };
    \end{tikzpicture}

    Answer \theqcounter: * What is the Common Type System (CTS)?%
  }

  {\footnotesize
    The CTS defines all data types and programming constructs supported by the CLR. It ensures that objects written in different .NET languages can interoperate safely by adhering to a common type specification.
  }
\end{frame}

% 5 ── What is the CLS?
\QuestionSlide[\CategoryBadge[MemoryColor!20]{Types \& Memory}]
  {* What is the Common Language Specification (CLS)?}
\begin{frame}[fragile]
  \frametitle{%
    \begin{tikzpicture}[remember picture,overlay]
      \node[anchor=north east,xshift=-0.4cm,yshift=-0.4cm] at (current page.north east) {
        \CategoryBadge[MemoryColor!20]{Types \& Memory}
      };
    \end{tikzpicture}

    Answer \theqcounter: * What is the Common Language Specification (CLS)?%
  }

  {\footnotesize
    The CLS is a subset of the CTS that defines rules and conventions (naming, visibility, type usage) which all .NET languages must follow to guarantee interoperability.
  }
\end{frame}

% 6 ── What is an immutable object?
\QuestionSlide[\CategoryBadge[MemoryColor!20]{Types \& Memory}]
  {* What does it mean that an object is immutable?}
\begin{frame}[fragile]
  \frametitle{%
    \begin{tikzpicture}[remember picture,overlay]
      \node[anchor=north east,xshift=-0.4cm,yshift=-0.4cm] at (current page.north east) {
        \CategoryBadge[MemoryColor!20]{Types \& Memory}
      };
    \end{tikzpicture}

    Answer \theqcounter: * What does it mean that an object is immutable?%
  }

  {\footnotesize
    An \textbf{immutable object} cannot change state after construction. All its fields are set once (typically via constructor) with no setters exposed, making instances inherently thread-safe and predictable.
  }

  \begin{minted}{csharp}
public class ImmutableUser
{
    public string Name { get; }

    public ImmutableUser(string name)
    {
        Name = name;
    }
}

// var user = new ImmutableUser("Alice");
// user.Name = "Bob"; // compile-time error
  \end{minted}
\end{frame}

% 7 ── Stack vs Heap
\QuestionSlide[\CategoryBadge[MemoryColor!20]{Types \& Memory}]
  {** What is stack vs heap allocation?}
\begin{frame}[fragile]
  \frametitle{%
    \begin{tikzpicture}[remember picture,overlay]
      \node[anchor=north east,xshift=-0.4cm,yshift=-0.4cm] at (current page.north east) {
        \CategoryBadge[MemoryColor!20]{Types \& Memory}
      };
    \end{tikzpicture}

    Answer \theqcounter: ** What is stack vs heap allocation?%
  }

  {\footnotesize
    The stack is a fast LIFO region for value types and local method data. The heap is a managed region for reference types and objects whose lifetimes are controlled by the garbage collector.
  }

  \begin{minted}{csharp}
int x = 5;               // value on stack
string s = "hello";      // reference on stack, data on heap
Person p = new Person(); // reference on stack, object on heap
  \end{minted}
\end{frame}

% 8 ── Which types go where?
\QuestionSlide[\CategoryBadge[MemoryColor!20]{Types \& Memory}]
  {** Which types go on the stack vs the heap?}
\begin{frame}[fragile]
  \frametitle{%
    \begin{tikzpicture}[remember picture,overlay]
      \node[anchor=north east,xshift=-0.4cm,yshift=-0.4cm] at (current page.north east) {
        \CategoryBadge[MemoryColor!20]{Types \& Memory}
      };
    \end{tikzpicture}

    Answer \theqcounter: ** Which types go on the stack vs the heap?%
  }

  {\footnotesize
    \begin{itemize}
      \item \textbf{Value types} (e.g., \texttt{int}, \texttt{struct}) live on the stack when declared locally.
      \item \textbf{Reference types} (e.g., \texttt{class}, \texttt{string}) reside on the heap, with only their reference on the stack or inside another object.
    \end{itemize}
  }

  \begin{minted}{csharp}
struct Point { public int X, Y; }

void Method()
{
    Point pt = new Point();     // struct on stack
    Point[] arr = new Point[3]; // array+structs on heap
}
  \end{minted}
\end{frame}

% 9 ── Why is stack faster?
\QuestionSlide[\CategoryBadge[MemoryColor!20]{Types \& Memory}]
  {** Why is stack allocation faster than heap?}
\begin{frame}[fragile]
  \frametitle{%
    \begin{tikzpicture}[remember picture,overlay]
      \node[anchor=north east,xshift=-0.4cm,yshift=-0.4cm] at (current page.north east) {
        \CategoryBadge[MemoryColor!20]{Types \& Memory}
      };
    \end{tikzpicture}

    Answer \theqcounter: ** Why is stack allocation faster than heap?%
  }

  {\footnotesize
    Stack allocation is a simple pointer bump (push/pop). Heap allocation involves complex bookkeeping, fragmentation, and later garbage collection.
  }

  \begin{minted}{csharp}
void DoWork()
{
    int x = 10;               // stack—very fast
    var list = new List<int>(); // heap—slower, GC overhead
}
  \end{minted}
\end{frame}

% 10 ── Can structs be on the heap?
\QuestionSlide[\CategoryBadge[MemoryColor!20]{Types \& Memory}]
  {** Can structs be on the heap?}
\begin{frame}[fragile]
  \frametitle{%
    \begin{tikzpicture}[remember picture,overlay]
      \node[anchor=north east,xshift=-0.4cm,yshift=-0.4cm] at (current page.north east) {
        \CategoryBadge[MemoryColor!20]{Types \& Memory}
      };
    \end{tikzpicture}

    Answer \theqcounter: ** Can structs be on the heap?%
  }

  {\footnotesize
    Yes. When a struct is a field of a class or captured by a lambda, it’s stored on the heap as part of that object’s memory.
  }

  \begin{minted}{csharp}
struct Point { public int X, Y; }
class Container { public Point P; }

var c = new Container(); // Container and its Point live on the heap
  \end{minted}
\end{frame}

% 11 ── Where is a struct allocated?
\QuestionSlide[\CategoryBadge[MemoryColor!20]{Types \& Memory}]
  {** Where is a struct instance allocated—stack or heap?}
\begin{frame}[fragile]
  \frametitle{%
    \begin{tikzpicture}[remember picture,overlay]
      \node[anchor=north east,xshift=-0.4cm,yshift=-0.4cm] at (current page.north east) {
        \CategoryBadge[MemoryColor!20]{Types \& Memory}
      };
    \end{tikzpicture}

    Answer \theqcounter: ** Where is a struct instance allocated—stack or heap?%
  }

  {\footnotesize
    Allocation depends on context:
    \begin{itemize}
      \item Local variable → usually on the stack.
      \item Field of a class or captured by closure → on the heap.
    \end{itemize}
  }

  \begin{minted}{csharp}
struct Car { public string Model; public int Year; }

void Example()
{
    var car = new Car(); // car on stack
}

class Garage
{
    public Car StoredCar; // StoredCar on heap
}
  \end{minted}
\end{frame}

% 12 ── Can structs store class types?
\QuestionSlide[\CategoryBadge[MemoryColor!20]{Types \& Memory}]
  {** What type can be inside a struct—can struct store a class type?}
\begin{frame}[fragile]
  \frametitle{%
    \begin{tikzpicture}[remember picture,overlay]
      \node[anchor=north east,xshift=-0.4cm,yshift=-0.4cm] at (current page.north east) {
        \CategoryBadge[MemoryColor!20]{Types \& Memory}
      };
    \end{tikzpicture}

    Answer \theqcounter: ** What type can be inside a struct—can struct store a class type?%
  }

  {\footnotesize
    Structs can contain any field type, including reference types (classes). The struct remains a value type but holds references to heap-allocated objects.
  }

  \begin{minted}{csharp}
class Engine { public int HP; }

struct Car
{
    public string Model;   // reference
    public Engine Engine;  // reference
    public int Year;       // value
}
  \end{minted}
\end{frame}

% 13 ── Value vs Reference types
\QuestionSlide[\CategoryBadge[MemoryColor!20]{Types \& Memory}]
  {** What are value types and reference types?}
\begin{frame}[fragile]
  \frametitle{%
    \begin{tikzpicture}[remember picture,overlay]
      \node[anchor=north east,xshift=-0.4cm,yshift=-0.4cm] at (current page.north east) {
        \CategoryBadge[MemoryColor!20]{Types \& Memory}
      };
    \end{tikzpicture}

    Answer \theqcounter: ** What are value types and reference types?%
  }

  {\footnotesize
    \begin{itemize}
      \item \textbf{Value types}: stored directly (stack or inline), copied on assignment (e.g., \texttt{int}, \texttt{struct}).
      \item \textbf{Reference types}: stored on the heap with references on the stack; assignment copies the reference (e.g., \texttt{class}, \texttt{string}).
    \end{itemize}
  }
\end{frame}

% 14 ── Boxing and Unboxing
\QuestionSlide[\CategoryBadge[MemoryColor!20]{Types \& Memory}]
  {** What is boxing and unboxing?}
\begin{frame}[fragile]
  \frametitle{%
    \begin{tikzpicture}[remember picture,overlay]
      \node[anchor=north east,xshift=-0.4cm,yshift=-0.4cm] at (current page.north east) {
        \CategoryBadge[MemoryColor!20]{Types \& Memory}
      };
    \end{tikzpicture}

    Answer \theqcounter: ** What is boxing and unboxing?%
  }

  {\footnotesize
    Boxing wraps a value type into an \texttt{object} (heap allocation). Unboxing extracts the value type from the \texttt{object}. Both incur performance overhead.
  }

  \begin{minted}{csharp}
int number = 42;
object boxed = number;     // boxing
int unboxed = (int)boxed;  // unboxing
  \end{minted}
\end{frame}

% 15 ── Classes vs Structs
\QuestionSlide[\CategoryBadge[OOPColor!20]{Language Basics}]
  {** What are classes and structs?}
\begin{frame}[fragile]
  \frametitle{%
    \begin{tikzpicture}[remember picture,overlay]
      \node[anchor=north east,xshift=-0.4cm,yshift=-0.4cm] at (current page.north east) {
        \CategoryBadge[OOPColor!20]{Language Basics}
      };
    \end{tikzpicture}

    Answer \theqcounter: ** What are classes and structs?%
  }

  {\footnotesize
    \begin{itemize}
      \item \textbf{Class}: reference type, supports inheritance, allocated on the heap.
      \item \textbf{Struct}: value type, no inheritance (other than \texttt{ValueType}), allocated inline (stack or within containing object).
    \end{itemize}
  }
\end{frame}

% 16 ── Interfaces
\QuestionSlide[\CategoryBadge[OOPColor!20]{Language Basics}]
  {** What are interfaces?}
\begin{frame}[fragile]
  \frametitle{%
    \begin{tikzpicture}[remember picture,overlay]
      \node[anchor=north east,xshift=-0.4cm,yshift=-0.4cm] at (current page.north east) {
        \CategoryBadge[OOPColor!20]{Language Basics}
      };
    \end{tikzpicture}

    Answer \theqcounter: ** What are interfaces?%
  }

  {\footnotesize
    Interfaces declare method and property contracts without implementation. Classes or structs implement them to ensure specific behaviors.
  }
\end{frame}

% 17 ── Abstract Classes
\QuestionSlide[\CategoryBadge[OOPColor!20]{Language Basics}]
  {** What are abstract classes?}
\begin{frame}[fragile]
  \frametitle{%
    \begin{tikzpicture}[remember picture,overlay]
      \node[anchor=north east,xshift=-0.4cm,yshift=-0.4cm] at (current page.north east) {
        \CategoryBadge[OOPColor!20]{Language Basics}
      };
    \end{tikzpicture}

    Answer \theqcounter: ** What are abstract classes?%
  }

  {\footnotesize
    Abstract classes can define both abstract (no body) and concrete members. They cannot be instantiated directly and serve as base types for derived classes.
  }
\end{frame}

% 18 ── Properties
\QuestionSlide[\CategoryBadge[OOPColor!20]{Language Basics}]
  {** What are properties?}
\begin{frame}[fragile]
  \frametitle{%
    \begin{tikzpicture}[remember picture,overlay]
      \node[anchor=north east,xshift=-0.4cm,yshift=-0.4cm] at (current page.north east) {
        \CategoryBadge[OOPColor!20]{Language Basics}
      };
    \end{tikzpicture}

    Answer \theqcounter: ** What are properties?%
  }

  {\footnotesize
    Properties encapsulate fields with \texttt{get}/\texttt{set} accessors, enabling validation, lazy loading, and encapsulation while preserving field-like syntax.
  }
\end{frame}

% 19 ── Indexers
\QuestionSlide[\CategoryBadge[OOPColor!20]{Language Basics}]
  {** What are indexers?}
\begin{frame}[fragile]
  \frametitle{%
    \begin{tikzpicture}[remember picture,overlay]
      \node[anchor=north east,xshift=-0.4cm,yshift=-0.4cm] at (current page.north east) {
        \CategoryBadge[OOPColor!20]{Language Basics}
      };
    \end{tikzpicture}

    Answer \theqcounter: ** What are indexers?%
  }

  {\footnotesize
    Indexers allow objects to be accessed like arrays via the \texttt{this[]} syntax, routing get/set through custom logic.
  }

  \begin{minted}{csharp}
public class NameBook
{
    private Dictionary<int,string> names = new();

    public string this[int index]
    {
        get => names.TryGetValue(index, out var n) ? n : "Unknown";
        set => names[index] = value;
    }
}

var book = new NameBook();
book[1] = "Alice";
Console.WriteLine(book[1]); // Alice
  \end{minted}
\end{frame}

% 20 ── Methods & Overloading
\QuestionSlide[\CategoryBadge[OOPColor!20]{Language Basics}]
  {** What are methods and method overloading?}
\begin{frame}[fragile]
  \frametitle{%
    \begin{tikzpicture}[remember picture,overlay]
      \node[anchor=north east,xshift=-0.4cm,yshift=-0.4cm] at (current page.north east) {
        \CategoryBadge[OOPColor!20]{Language Basics}
      };
    \end{tikzpicture}

    Answer \theqcounter: ** What are methods and method overloading?%
  }

  {\footnotesize
    Methods encapsulate behavior. Overloading lets you define multiple methods with the same name but different parameter lists or signatures.
  }
\end{frame}

% 21 ── Constructors & Destructors
\QuestionSlide[\CategoryBadge[OOPColor!20]{Language Basics}]
  {** What are constructors and destructors?}
\begin{frame}[fragile]
  \frametitle{%
    \begin{tikzpicture}[remember picture,overlay]
      \node[anchor=north east,xshift=-0.4cm,yshift=-0.4cm] at (current page.north east) {
        \CategoryBadge[OOPColor!20]{Language Basics}
      };
    \end{tikzpicture}

    Answer \theqcounter: ** What are constructors and destructors?%
  }

  {\footnotesize
    \begin{itemize}
      \item \textbf{Constructor}: special method to initialize new object instances.
      \item \textbf{Destructor (finalizer)}: cleanup logic run by the GC before reclaiming the object (\texttt{\textasciitilde ClassName()}).
    \end{itemize}
  }
\end{frame}

% 22 ── Garbage Collection
\QuestionSlide[\CategoryBadge[MemoryColor!20]{Types \& Memory}]
  {** What is garbage collection?}
\begin{frame}[fragile]
  \frametitle{%
    \begin{tikzpicture}[remember picture,overlay]
      \node[anchor=north east,xshift=-0.4cm,yshift=-0.4cm] at (current page.north east) {
        \CategoryBadge[MemoryColor!20]{Types \& Memory}
      };
    \end{tikzpicture}

    Answer \theqcounter: ** What is garbage collection?%
  }

  {\footnotesize
    Garbage collection is the CLR’s automatic memory management. It periodically identifies unreachable objects and reclaims their memory, preventing leaks and reducing the need for manual deallocation.
  }

  \begin{minted}{csharp}
class Demo
{
    public void CreateObjects()
    {
        var buffer = new byte[1024*1024]; // 1MB array
    }
}
// 'buffer' is collected when out of scope and GC runs
  \end{minted}
\end{frame}


\hypertarget{sec2}{}
\section{C\# Language: Intermediate}
% 1 ── Dictionary<TKey, TValue>
\QuestionSlide[\CategoryBadge[CollColor!20]{Collections}]
  {** What is \texttt{Dictionary<TKey, TValue>} in C\#?}
\begin{frame}[fragile]
  \frametitle{%
    \begin{tikzpicture}[remember picture,overlay]
      \node[anchor=north east,xshift=-0.4cm,yshift=-0.4cm] at (current page.north east) {
        \CategoryBadge[CollColor!20]{Collections}
      };
    \end{tikzpicture}
    Answer \theqcounter: ** What is \texttt{Dictionary<TKey, TValue>} in C\#?%
  }

  {\footnotesize
    A \texttt{Dictionary<TKey, TValue>} is a generic collection in C\# that stores key–value pairs and provides fast lookup by key. It uses a hash table internally, offering average \(O(1)\) time complexity for \texttt{Add}, \texttt{Remove}, and \texttt{TryGetValue} operations. Keys must be unique and non-null (for reference types), and must implement a stable \texttt{GetHashCode()} and \texttt{Equals()}.
  }

  \begin{minted}{csharp}
var dict = new Dictionary<string, int>();
dict["apple"] = 3;
dict["banana"] = 5;
Console.WriteLine(dict["apple"]); // 3
  \end{minted}
\end{frame}

% 2 ── Handling keys
\QuestionSlide[\CategoryBadge[CollColor!20]{Collections}]
  {** How does a dictionary handle keys in C\#?}
\begin{frame}[fragile]
  \frametitle{%
    \begin{tikzpicture}[remember picture,overlay]
      \node[anchor=north east,xshift=-0.4cm,yshift=-0.4cm] at (current page.north east) {
        \CategoryBadge[CollColor!20]{Collections}
      };
    \end{tikzpicture}
    Answer \theqcounter: ** How does a dictionary handle keys in C\#?%
  }

  {\footnotesize
    When inserting or retrieving a value, \texttt{Dictionary} calls the key’s \texttt{GetHashCode()} to locate the bucket and then \texttt{Equals()} to confirm equality. To work reliably, keys should be immutable and must not change their hash code while stored in the dictionary.
  }

  \begin{minted}{csharp}
// Good key implementation
public class MyKey
{
    public int Id { get; }
    public MyKey(int id) => Id = id;
    public override int GetHashCode() => Id.GetHashCode();
    public override bool Equals(object obj) =>
        obj is MyKey other && other.Id == Id;
}

var dict = new Dictionary<MyKey, string>();
  \end{minted}
\end{frame}

% 3 ── Complexity of operations
\QuestionSlide[\CategoryBadge[CollColor!20]{Collections}]
  {** What are the time and space complexities of \texttt{Dictionary<TKey, TValue>}?}
\begin{frame}[fragile]
  \frametitle{%
    \begin{tikzpicture}[remember picture,overlay]
      \node[anchor=north east,xshift=-0.4cm,yshift=-0.4cm] at (current page.north east) {
        \CategoryBadge[CollColor!20]{Collections}
      };
    \end{tikzpicture}
    Answer \theqcounter: ** What are the time and space complexities of \texttt{Dictionary<TKey, TValue>}?%
  }

  {\footnotesize
    \textbf{Average-case:}
    \begin{itemize}
      \item \texttt{Add}, \texttt{Remove}, \texttt{TryGetValue} – \(O(1)\)
      \item Iteration – \(O(n)\)
    \end{itemize}
    \textbf{Worst-case:}
    \begin{itemize}
      \item If many keys collide, operations degrade to \(O(n)\)
    \end{itemize}
    \textbf{Space:}
    \begin{itemize}
      \item \(O(n)\) for entries plus overhead for buckets and collision resolution
    \end{itemize}
  }

  \begin{minted}{csharp}
// Bad hash example leading to worst-case
class BadKey
{
    public override int GetHashCode() => 1;
    public override bool Equals(object obj) => true;
}
  \end{minted}
\end{frame}

% 4 ── Best practices
\QuestionSlide[\CategoryBadge[CollColor!20]{Collections}]
  {** What are best practices when using \texttt{Dictionary<TKey, TValue>}?}
\begin{frame}[fragile]
  \frametitle{%
    \begin{tikzpicture}[remember picture,overlay]
      \node[anchor=north east,xshift=-0.4cm,yshift=-0.4cm] at (current page.north east) {
        \CategoryBadge[CollColor!20]{Collections}
      };
    \end{tikzpicture}
    Answer \theqcounter: ** What are best practices when using \texttt{Dictionary<TKey, TValue>}?%
  }

  {\footnotesize
    \begin{itemize}
      \item Use immutable keys (e.g., strings, readonly structs).
      \item Prefer \texttt{TryGetValue} to avoid exceptions.
      \item Specify initial capacity when known to reduce resizing.
      \item Use \texttt{ContainsKey} before index access if unsure.
    \end{itemize}
  }

  \begin{minted}{csharp}
if (dict.TryGetValue("banana", out int qty))
    Console.WriteLine($"Qty: {qty}");
else
    Console.WriteLine("Not found");
  \end{minted}
\end{frame}

% 5 ── Average-case complexity
\QuestionSlide[\CategoryBadge[CollColor!20]{Collections}]
  {** What is the average time complexity of dictionary operations in C\#?}
\begin{frame}[fragile]
  \frametitle{%
    \begin{tikzpicture}[remember picture,overlay]
      \node[anchor=north east,xshift=-0.4cm,yshift=-0.4cm] at (current page.north east) {
        \CategoryBadge[CollColor!20]{Collections}
      };
    \end{tikzpicture}
    Answer \theqcounter: ** What is the average time complexity of dictionary operations in C\#?%
  }

  {\footnotesize
    On average, \texttt{Add}, \texttt{Remove}, and \texttt{TryGetValue} all run in \(O(1)\), assuming a good hash distribution and low collision rate.
  }

  \begin{minted}{csharp}
dict.Add("key1", 10);
dict.TryGetValue("key1", out int value);
dict.Remove("key1");
  \end{minted}
\end{frame}

% 6 ── Worst-case complexity
\QuestionSlide[\CategoryBadge[CollColor!20]{Collections}]
  {** What is the worst-case time complexity for dictionary operations?}
\begin{frame}[fragile]
  \frametitle{%
    \begin{tikzpicture}[remember picture,overlay]
      \node[anchor=north east,xshift=-0.4cm,yshift=-0.4cm] at (current page.north east) {
        \CategoryBadge[CollColor!20]{Collections}
      };
    \end{tikzpicture}
    Answer \theqcounter: ** What is the worst-case time complexity for dictionary operations?%
  }

  {\footnotesize
    In the worst case (e.g., all keys collide), operations degrade to \(O(n)\), since the dictionary must scan a bucket list of all entries.
  }

  \begin{minted}{csharp}
// Example of poor key:
class BadKey { public override int GetHashCode() => 1; public override bool Equals(object o) => true; }
  \end{minted}
\end{frame}

% 7 ── Iteration complexity
\QuestionSlide[\CategoryBadge[CollColor!20]{Collections}]
  {** What is the time complexity of iterating over a dictionary?}
\begin{frame}[fragile]
  \frametitle{%
    \begin{tikzpicture}[remember picture,overlay]
      \node[anchor=north east,xshift=-0.4cm,yshift=-0.4cm] at (current page.north east) {
        \CategoryBadge[CollColor!20]{Collections}
      };
    \end{tikzpicture}
    Answer \theqcounter: ** What is the time complexity of iterating over a dictionary?%
  }

  {\footnotesize
    Iteration runs in \(O(n)\), where \(n\) is the number of entries. The order isn’t sorted and reflects internal bucket organization.
  }

  \begin{minted}{csharp}
foreach (var kvp in dict)
    Console.WriteLine($"{kvp.Key} = {kvp.Value}");
  \end{minted}
\end{frame}

% 8 ── Space complexity
\QuestionSlide[\CategoryBadge[CollColor!20]{Collections}]
  {** What is the space complexity of a dictionary in C\#?}
\begin{frame}[fragile]
  \frametitle{%
    \begin{tikzpicture}[remember picture,overlay]
      \node[anchor=north east,xshift=-0.4cm,yshift=-0.4cm] at (current page.north east) {
        \CategoryBadge[CollColor!20]{Collections}
      };
    \end{tikzpicture}
    Answer \theqcounter: ** What is the space complexity of a dictionary in C\#?%
  }

  {\footnotesize
    Space usage is \(O(n)\) for \(n\) entries, plus overhead for:
    \begin{itemize}
      \item Buckets array (resized as needed)
      \item Entry objects and collision chains
    \end{itemize}
  }

  \begin{minted}{csharp}
var dict = new Dictionary<string, int>(capacity: 1000);
  \end{minted}
\end{frame}

% 9 ── Ensuring constant-time
\QuestionSlide[\CategoryBadge[CollColor!20]{Collections}]
  {** How do you ensure dictionary operations stay constant-time?}
\begin{frame}[fragile]
  \frametitle{%
    \begin{tikzpicture}[remember picture,overlay]
      \node[anchor=north east,xshift=-0.4cm,yshift=-0.4cm] at (current page.north east) {
        \CategoryBadge[CollColor!20]{Collections}
      };
    \end{tikzpicture}
    Answer \theqcounter: ** How do you ensure dictionary operations stay constant-time?%
  }

  {\footnotesize
    \begin{itemize}
      \item Use immutable, well-distributed keys (e.g., strings, GUIDs).
      \item Avoid poor \texttt{GetHashCode()} implementations.
      \item Pre-size the dictionary if entry count is known.
      \item Never mutate keys after insertion.
    \end{itemize}
  }

  \begin{minted}{csharp}
var dict = new Dictionary<Guid, string>();
Guid id = Guid.NewGuid();
dict[id] = "session";
  \end{minted}
\end{frame}

% 10 ── Reflection capabilities
\QuestionSlide[\CategoryBadge[MetaColor!20]{Metaprogramming}]
  {** What can you do with reflection in C\#?}
\begin{frame}[fragile]
  \frametitle{%
    \begin{tikzpicture}[remember picture,overlay]
      \node[anchor=north east,xshift=-0.4cm,yshift=-0.4cm] at (current page.north east) {
        \CategoryBadge[MetaColor!20]{Metaprogramming}
      };
    \end{tikzpicture}
    Answer \theqcounter: ** What can you do with reflection in C\#?%
  }

  {\footnotesize
    Reflection allows you to:
    \begin{itemize}
      \item Inspect types, methods, properties, and fields.
      \item Invoke methods dynamically.
      \item Read/write fields and properties.
      \item Access custom attributes at runtime.
    \end{itemize}
    Common uses include serialization, plugin systems, dependency injection, and dynamic API discovery.
  }

  \begin{minted}{csharp}
var type = typeof(Person);
foreach (var method in type.GetMethods())
    Console.WriteLine(method.Name);

var attrs = type.GetProperty("Name")
                .GetCustomAttributes(typeof(ObsoleteAttribute), false);
  \end{minted}
\end{frame}

% 11 ── Runtime reflection
\QuestionSlide[\CategoryBadge[MetaColor!20]{Metaprogramming}]
  {** What is runtime reflection in C\#?}
\begin{frame}[fragile]
  \frametitle{%
    \begin{tikzpicture}[remember picture,overlay]
      \node[anchor=north east,xshift=-0.4cm,yshift=-0.4cm] at (current page.north east) {
        \CategoryBadge[MetaColor!20]{Metaprogramming}
      };
    \end{tikzpicture}
    Answer \theqcounter: ** What is runtime reflection in C\#?%
  }

  {\footnotesize
    Runtime reflection is inspecting and interacting with a program’s metadata and types during execution. It uses \texttt{System.Reflection} to examine assemblies, types, members, and attributes dynamically, at the cost of performance and compile-time safety.
  }

  \begin{minted}{csharp}
Type t = typeof(MyClass);
foreach (var prop in t.GetProperties())
    Console.WriteLine($"{prop.Name}: {prop.PropertyType}");
  \end{minted}
\end{frame}

% 12 ── Reflection vs. runtime reflection
\QuestionSlide[\CategoryBadge[MetaColor!20]{Metaprogramming}]
  {** What is the difference between reflection and runtime reflection?}
\begin{frame}[fragile]
  \frametitle{%
    \begin{tikzpicture}[remember picture,overlay]
      \node[anchor=north east,xshift=-0.4cm,yshift=-0.4cm] at (current page.north east) {
        \CategoryBadge[MetaColor!20]{Metaprogramming}
      };
    \end{tikzpicture}
    Answer \theqcounter: ** What is the difference between reflection and runtime reflection?%
  }

  {\footnotesize
    “Reflection” refers broadly to \texttt{System.Reflection} APIs for inspecting metadata. “Runtime reflection” emphasizes that this inspection occurs during program execution, highlighting dynamic behavior and performance overhead.
  }

  \begin{minted}{csharp}
var method = typeof(MyClass).GetMethod("Execute");
method.Invoke(Activator.CreateInstance(typeof(MyClass)), null);
  \end{minted}
\end{frame}

% 13 ── Span<T>
\QuestionSlide[\CategoryBadge[PerfColor!20]{Performance}]
  {** What is \texttt{Span<T>}?}
\begin{frame}[fragile]
  \frametitle{%
    \begin{tikzpicture}[remember picture,overlay]
      \node[anchor=north east,xshift=-0.4cm,yshift=-0.4cm] at (current page.north east) {
        \CategoryBadge[PerfColor!20]{Performance}
      };
    \end{tikzpicture}
    Answer \theqcounter: ** What is \texttt{Span<T>}?%
  }

  {\footnotesize
    \texttt{Span<T>} is a stack-only struct that provides a safe, memory-efficient slice over contiguous memory (arrays, strings, or unmanaged buffers) without allocations. It’s ideal for performance-critical parsing or buffer manipulation.
  }

  \begin{minted}{csharp}
int[] numbers = {1,2,3,4,5};
Span<int> slice = numbers.AsSpan(1,3); // [2,3,4]
slice[0] = 42;
Console.WriteLine(numbers[1]); // 42
  \end{minted}
\end{frame}

% 14 ── Code weaving
\QuestionSlide[\CategoryBadge[MetaColor!20]{Metaprogramming}]
  {** What is code weaving in .NET?}
\begin{frame}[fragile]
  \frametitle{%
    \begin{tikzpicture}[remember picture,overlay]
      \node[anchor=north east,xshift=-0.4cm,yshift=-0.4cm] at (current page.north east) {
        \CategoryBadge[MetaColor!20]{Metaprogramming}
      };
    \end{tikzpicture}
    Answer \theqcounter: ** What is code weaving in .NET?%
  }

  {\footnotesize
    Code weaving is a post-compile technique where tools (e.g., Fody) inject or modify IL in assemblies for cross-cutting concerns (logging, validation, metrics) without altering source code, though it can obscure behavior and complicate debugging.
  }

  \begin{minted}{csharp}
// Fody injects INotifyPropertyChanged
public class Person : INotifyPropertyChanged
{
    public string Name { get; set; }
}
  \end{minted}
\end{frame}

% 15 ── Cross-cutting concerns
\QuestionSlide[\CategoryBadge[ArchColor!20]{Architecture}]
  {** What are cross-cutting concerns?}
\begin{frame}[fragile]
  \frametitle{%
    \begin{tikzpicture}[remember picture,overlay]
      \node[anchor=north east,xshift=-0.4cm,yshift=-0.4cm] at (current page.north east) {
        \CategoryBadge[ArchColor!20]{Architecture}
      };
    \end{tikzpicture}
    Answer \theqcounter: ** What are cross-cutting concerns?%
  }

  {\footnotesize
    Cross-cutting concerns—such as logging, validation, error handling, caching, or security—affect multiple layers of an application. They're centralized via middleware, AOP frameworks, or source generators to avoid duplication and maintain separation of concerns.
  }

  \begin{minted}{csharp}
// ASP.NET Core logging middleware example
public class LoggingMiddleware
{
    private readonly RequestDelegate _next;
    public LoggingMiddleware(RequestDelegate next) => _next = next;

    public async Task Invoke(HttpContext ctx)
    {
        Console.WriteLine($"Request: {ctx.Request.Path}");
        await _next(ctx);
        Console.WriteLine($"Response: {ctx.Response.StatusCode}");
    }
}
  \end{minted}
\end{frame}

% 16 ── Source generators
\QuestionSlide[\CategoryBadge[MetaColor!20]{Metaprogramming}]
  {** What are source generators?}
\begin{frame}[fragile]
  \frametitle{%
    \begin{tikzpicture}[remember picture,overlay]
      \node[anchor=north east,xshift=-0.4cm,yshift=-0.4cm] at (current page.north east) {
        \CategoryBadge[MetaColor!20]{Metaprogramming}
      };
    \end{tikzpicture}
    Answer \theqcounter: ** What are source generators?%
  }

  {\footnotesize
    Source generators run at compile time to analyze user code and emit additional C\# files. They reduce boilerplate and eliminate runtime reflection, enabling high-performance metaprogramming (e.g., JSON serializers, DI scaffolding).
  }

  \begin{minted}{csharp}
[Generator]
public class HelloWorldGenerator : ISourceGenerator
{
    public void Initialize(GeneratorInitializationContext ctx) { }
    public void Execute(GeneratorExecutionContext ctx)
    {
        ctx.AddSource("Hello.g.cs", @"
            public static class Hello
            {
                public static void SayHi() =>
                    Console.WriteLine(\"Hello from generated code!\");
            }");
    }
}
  \end{minted}
\end{frame}

% 17 ── params keyword
\QuestionSlide[\CategoryBadge[LangColor!20]{Language Feature}]
  {** What is the \texttt{params} keyword?}
\begin{frame}[fragile]
  \frametitle{%
    \begin{tikzpicture}[remember picture,overlay]
      \node[anchor=north east,xshift=-0.4cm,yshift=-0.4cm] at (current page.north east) {
        \CategoryBadge[LangColor!20]{Language Feature}
      };
    \end{tikzpicture}
    Answer \theqcounter: ** What is the \texttt{params} keyword?%
  }

  {\footnotesize
    The \texttt{params} keyword allows a method to accept a variable number of arguments as an array.  
    E.g., \texttt{void Foo(params int[] nums)} can be called as \texttt{Foo(1,2,3)}.
  }

  \begin{minted}{csharp}
void Foo(params int[] nums)
{
    foreach (var n in nums) Console.WriteLine(n);
}

Foo(1,2,3); // outputs 1, 2, 3
  \end{minted}
\end{frame}

% 18 ── Iterators & yield
\QuestionSlide[\CategoryBadge[LangColor!20]{Language Feature}]
  {** What are iterators and the \texttt{yield} keyword?}
\begin{frame}[fragile]
  \frametitle{%
    \begin{tikzpicture}[remember picture,overlay]
      \node[anchor=north east,xshift=-0.4cm,yshift=-0.4cm] at (current page.north east) {
        \CategoryBadge[LangColor!20]{Language Feature}
      };
    \end{tikzpicture}
    Answer \theqcounter: ** What are iterators and the \texttt{yield} keyword?%
  }

  {\footnotesize
    Iterators with \texttt{yield return} allow lazy, streaming generation of sequences without intermediate collections.
  }

  \begin{minted}{csharp}
IEnumerable<int> CountTo(int n)
{
  for (int i = 1; i <= n; i++)
    yield return i;
}

foreach (var x in CountTo(3))
    Console.WriteLine(x); // 1,2,3
  \end{minted}
\end{frame}

% 19 ── Partial classes & methods
\QuestionSlide[\CategoryBadge[LangColor!20]{Language Feature}]
  {** What are partial classes and methods?}
\begin{frame}[fragile]
  \frametitle{%
    \begin{tikzpicture}[remember picture,overlay]
      \node[anchor=north east,xshift=-0.4cm,yshift=-0.4cm] at (current page.north east) {
        \CategoryBadge[LangColor!20]{Language Feature}
      };
    \end{tikzpicture}
    Answer \theqcounter: ** What are partial classes and methods?%
  }

  {\footnotesize
    \texttt{partial class} lets you split a class definition across files.  
    \texttt{partial method} declares a hook without implementation; it’s removed if never implemented.
  }

  \begin{minted}{csharp}
// File A
public partial class MyClass
{
    partial void OnInit();
    public void Init() => OnInit();
}

// File B
public partial class MyClass
{
    partial void OnInit() => Console.WriteLine("Initialized");
}
  \end{minted}
\end{frame}

% 20 ── Anonymous types
\QuestionSlide[\CategoryBadge[LangColor!20]{Language Feature}]
  {** What are anonymous types?}
\begin{frame}[fragile]
  \frametitle{%
    \begin{tikzpicture}[remember picture,overlay]
      \node[anchor=north east,xshift=-0.4cm,yshift=-0.4cm] at (current page.north east) {
        \CategoryBadge[LangColor!20]{Language Feature}
      };
    \end{tikzpicture}
    Answer \theqcounter: ** What are anonymous types?%
  }

  {\footnotesize
    Anonymous types are compiler-generated reference types with readonly properties inferred from initializer syntax, ideal for LINQ projections without declaring a class.
  }

  \begin{minted}{csharp}
var person = new { Name = "Alice", Age = 30 };
Console.WriteLine(person.Name); // "Alice"

var users = people.Select(u => new { u.Id, u.Name });
  \end{minted}
\end{frame}

% 21 ── Pattern matching
\QuestionSlide[\CategoryBadge[LangColor!20]{Language Feature}]
  {** What is pattern matching?}
\begin{frame}[fragile]
  \frametitle{%
    \begin{tikzpicture}[remember picture,overlay]
      \node[anchor=north east,xshift=-0.4cm,yshift=-0.4cm] at (current page.north east) {
        \CategoryBadge[LangColor!20]{Language Feature}
      };
    \end{tikzpicture}
    Answer \theqcounter: ** What is pattern matching?%
  }

  {\footnotesize
    Pattern matching enhances \texttt{is} and \texttt{switch} with rich syntax for type, constant, property, positional, relational, and logical patterns, simplifying complex conditional logic.
  }

  \begin{minted}{csharp}
public static string Classify(object o) =>
  o switch
  {
    null => "No value",
    int i when i > 0 => "Positive",
    string s => $"Text({s.Length})",
    Person { Age: >= 18 } p => $"{p.Name} is adult",
    (int x,int y) => $"Point({x},{y})",
    _ => "Unknown"
  };
  \end{minted}
\end{frame}

% 22 ── Default interface implementations
\QuestionSlide[\CategoryBadge[LangColor!20]{Language Feature}]
  {** What are default interface implementations?}
\begin{frame}[fragile]
  \frametitle{%
    \begin{tikzpicture}[remember picture,overlay]
      \node[anchor=north east,xshift=-0.4cm,yshift=-0.4cm] at (current page.north east) {
        \CategoryBadge[LangColor!20]{Language Feature}
      };
    \end{tikzpicture}
    Answer \theqcounter: ** What are default interface implementations?%
  }

  {\footnotesize
    Introduced in C\# 8, default implementations let interfaces provide method bodies, enabling API evolution without breaking existing implementers.
  }

  \begin{minted}{csharp}
public interface ICustomer
{
  string Name { get; }
  DateTime Joined { get; }
  decimal GetDiscount() => Joined < DateTime.UtcNow.AddYears(-2)
    ? 0.10m : 0m;
}

public class Customer : ICustomer
{
  public string Name { get; }
  public DateTime Joined { get; }
}
  \end{minted}
\end{frame}

% 23 ── Record types
\QuestionSlide[\CategoryBadge[LangColor!20]{Language Feature}]
  {** What are record types?}
\begin{frame}[fragile]
  \frametitle{%
    \begin{tikzpicture}[remember picture,overlay]
      \node[anchor=north east,xshift=-0.4cm,yshift=-0.4cm] at (current page.north east) {
        \CategoryBadge[LangColor!20]{Language Feature}
      };
    \end{tikzpicture}
    Answer \theqcounter: ** What are record types?%
  }

  {\footnotesize
    Records are immutable reference types for value-based data models. They support concise syntax, built-in \texttt{Equals}, \texttt{GetHashCode}, \texttt{ToString}, and non-destructive mutation via \texttt{with}.
  }

  \begin{minted}{csharp}
public record Person(string Name, int Age);
var p1 = new Person("Alice",30);
var p2 = p1 with { Age = 31 };
Console.WriteLine(p1 == p2); // False
  \end{minted}
\end{frame}

% 24 ── Value-based equality
\QuestionSlide[\CategoryBadge[LangColor!20]{Language Feature}]
  {** What is value-based equality in record types?}
\begin{frame}[fragile]
  \frametitle{%
    \begin{tikzpicture}[remember picture,overlay]
      \node[anchor=north east,xshift=-0.4cm,yshift=-0.4cm] at (current page.north east) {
        \CategoryBadge[LangColor!20]{Language Feature}
      };
    \end{tikzpicture}
    Answer \theqcounter: ** What is value-based equality in record types?%
  }

  {\footnotesize
    Two record instances are equal if all their properties have equal values, regardless of reference identity, unlike classes which default to reference equality.
  }

  \begin{minted}{csharp}
var p1 = new Person("Alice",30);
var p2 = new Person("Alice",30);
Console.WriteLine(p1 == p2); // True
  \end{minted}
\end{frame}

% 25 ── Concise syntax
\QuestionSlide[\CategoryBadge[LangColor!20]{Language Feature}]
  {** What does concise syntax mean for record types?}
\begin{frame}[fragile]
  \frametitle{%
    \begin{tikzpicture}[remember picture,overlay]
      \node[anchor=north east,xshift=-0.4cm,yshift=-0.4cm] at (current page.north east) {
        \CategoryBadge[LangColor!20]{Language Feature}
      };
    \end{tikzpicture}
    Answer \theqcounter: ** What does concise syntax mean for record types?%
  }

  {\footnotesize
    Records allow defining data carriers, constructors, and value-based behaviors (e.g., \texttt{ToString}, equality) in a single line, reducing boilerplate.
  }

  \begin{minted}{csharp}
// Full class:
public class Person
{
  public string Name { get; }
  public int Age { get; }
  public Person(string name,int age)=> (Name,Age)=(name,age);
}

// Record:
public record Person(string Name,int Age);
  \end{minted}
\end{frame}

% 26 ── With-expressions
\QuestionSlide[\CategoryBadge[LangColor!20]{Language Feature}]
  {** What is non-destructive mutation using \texttt{with}?}
\begin{frame}[fragile]
  \frametitle{%
    \begin{tikzpicture}[remember picture,overlay]
      \node[anchor=north east,xshift=-0.4cm,yshift=-0.4cm] at (current page.north east) {
        \CategoryBadge[LangColor!20]{Language Feature}
      };
    \end{tikzpicture}
    Answer \theqcounter: ** What is non-destructive mutation using \texttt{with}?%
  }

  {\footnotesize
    The \texttt{with} expression creates a new record instance with specified property changes, leaving the original unchanged.
  }

  \begin{minted}{csharp}
var original = new Person("Alice",30);
var updated = original with { Age = 31 };
Console.WriteLine(original.Age); // 30
Console.WriteLine(updated.Age);  // 31
  \end{minted}
\end{frame}

% 27 ── dynamic type
\QuestionSlide[\CategoryBadge[LangColor!20]{Language Feature}]
  {** What is the \texttt{dynamic} type?}
\begin{frame}[fragile]
  \frametitle{%
    \begin{tikzpicture}[remember picture,overlay]
      \node[anchor=north east,xshift=-0.4cm,yshift=-0.4cm] at (current page.north east) {
        \CategoryBadge[LangColor!20]{Language Feature}
      };
    \end{tikzpicture}
    Answer \theqcounter: ** What is the \texttt{dynamic} type?%
  }

  {\footnotesize
    \texttt{dynamic} bypasses compile-time type checking, resolving member calls at runtime. Useful for COM interop, dynamic languages, or loosely typed data, but error-prone if misused.
  }

  \begin{minted}{csharp}
dynamic obj = "hello";
Console.WriteLine(obj.Length); // 5
obj = 123;
// Console.WriteLine(obj.Length); // runtime error
  \end{minted}
\end{frame}

% 28 ── Expression trees
\QuestionSlide[\CategoryBadge[MetaColor!20]{Metaprogramming}]
  {** What are expression trees?}
\begin{frame}[fragile]
  \frametitle{%
    \begin{tikzpicture}[remember picture,overlay]
      \node[anchor=north east,xshift=-0.4cm,yshift=-0.4cm] at (current page.north east) {
        \CategoryBadge[MetaColor!20]{Metaprogramming}
      };
    \end{tikzpicture}
    Answer \theqcounter: ** What are expression trees?%
  }

  {\footnotesize
    Expression trees are data structures representing code as a tree of \texttt{Expression} objects (e.g., \texttt{Expression<Func<T,bool>>}), used by LINQ providers to translate queries.
  }

  \begin{minted}{csharp}
using System.Linq.Expressions;
Expression<Func<int,bool>> isEven = x => x % 2 == 0;
Console.WriteLine(isEven.Body); // (x % 2) == 0
  \end{minted}
\end{frame}

% 29 ── ref vs out
\QuestionSlide[\CategoryBadge[LangColor!20]{Language Feature}]
  {** What is the difference between \texttt{ref} and \texttt{out} keywords?}
\begin{frame}[fragile]
  \frametitle{%
    \begin{tikzpicture}[remember picture,overlay]
      \node[anchor=north east,xshift=-0.4cm,yshift=-0.4cm] at (current page.north east) {
        \CategoryBadge[LangColor!20]{Language Feature}
      };
    \end{tikzpicture}
    Answer \theqcounter: ** What is the difference between \texttt{ref} and \texttt{out}?%
  }

  {\footnotesize
    \begin{itemize}
      \item \texttt{ref}: parameter must be initialized before call.
      \item \texttt{out}: parameter need not be initialized; must be assigned in method.
    \end{itemize}
  }

  \begin{minted}{csharp}
void M(ref int x)  { /* x must be set */ }
void N(out int x) { x = 42; }

int a = 1; M(ref a);
int b;     N(out b);
  \end{minted}
\end{frame}

% 30 ── ref, out, in modifiers
\QuestionSlide[\CategoryBadge[LangColor!20]{Language Feature}]
  {When would you use the \texttt{ref}, \texttt{out} and \texttt{in} modifiers?}
\begin{frame}[fragile]
  \frametitle{%
    \begin{tikzpicture}[remember picture,overlay]
      \node[anchor=north east,xshift=-0.4cm,yshift=-0.4cm] at (current page.north east) {
        \CategoryBadge[LangColor!20]{Language Feature}
      };
    \end{tikzpicture}
    Answer \theqcounter: When would you use \texttt{ref}, \texttt{out}, and \texttt{in}?%
  }

  {\footnotesize
    \begin{itemize}
      \item \texttt{ref}: read/write reference, variable must be initialized.
      \item \texttt{out}: write-only reference, must be assigned by callee.
      \item \texttt{in}: readonly reference for passing large structs without copying.
    \end{itemize}
  }

  \begin{minted}{csharp}
// ref: swap values
void Swap(ref int x, ref int y) { var t = x; x = y; y = t; }
// out: parse result
bool TryParse(string s, out int v) { return int.TryParse(s, out v); }
// in: pass large struct efficiently
void Process(in BigStruct data) { /* read-only */ }
  \end{minted}
\end{frame}

% 31 ── Delegates
\QuestionSlide[\CategoryBadge[LangColor!20]{Language Feature}]
  {** What are delegates?}
\begin{frame}[fragile]
  \frametitle{%
    \begin{tikzpicture}[remember picture,overlay]
      \node[anchor=north east,xshift=-0.4cm,yshift=-0.4cm] at (current page.north east) {
        \CategoryBadge[LangColor!20]{Language Feature}
      };
    \end{tikzpicture}
    Answer \theqcounter: ** What are delegates?%
  }

  {\footnotesize
    Delegates are type-safe function pointers encapsulating method references. They enable passing methods as parameters, support multicast invocation, and underpin events.
  }

  \begin{minted}{csharp}
public delegate void Notify(string msg);
var m = new Messenger();
m.OnNotify = msg => Console.WriteLine(msg);
m.Send("Hello");
  \end{minted}
\end{frame}

% 32 ── Events
\QuestionSlide[\CategoryBadge[LangColor!20]{Language Feature}]
  {** What are events?}
\begin{frame}[fragile]
  \frametitle{%
    \begin{tikzpicture}[remember picture,overlay]
      \node[anchor=north east,xshift=-0.4cm,yshift=-0.4cm] at (current page.north east) {
        \CategoryBadge[LangColor!20]{Language Feature}
      };
    \end{tikzpicture}
    Answer \theqcounter: ** What are events?%
  }

  {\footnotesize
    Events are special delegates enabling the publish–subscribe pattern. The \texttt{event} keyword restricts raising to the declaring class, ensuring encapsulation.
  }

  \begin{minted}{csharp}
public class Button
{
  public event EventHandler Click;
  protected void OnClick() => Click?.Invoke(this, EventArgs.Empty);
}
  \end{minted}
\end{frame}

% 33 ── Anonymous methods
\QuestionSlide[\CategoryBadge[LangColor!20]{Language Feature}]
  {** What is an anonymous method?}
\begin{frame}[fragile]
  \frametitle{%
    \begin{tikzpicture}[remember picture,overlay]
      \node[anchor=north east,xshift=-0.4cm,yshift=-0.4cm] at (current page.north east) {
        \CategoryBadge[LangColor!20]{Language Feature}
      };
    \end{tikzpicture}
    Answer \theqcounter: ** What is an anonymous method?%
  }

  {\footnotesize
    An anonymous method defines an inline delegate with the \texttt{delegate} keyword, useful for quick, short-lived functions and closures.
  }

  \begin{minted}{csharp}
delegate void Greet(string name);
Greet g = delegate(string n) { Console.WriteLine($"Hello, {n}!"); };
g("Alice");
  \end{minted}
\end{frame}

% 34 ── Lambda expressions
\QuestionSlide[\CategoryBadge[LangColor!20]{Language Feature}]
  {** What are lambda expressions?}
\begin{frame}[fragile]
  \frametitle{%
    \begin{tikzpicture}[remember picture,overlay]
      \node[anchor=north east,xshift=-0.4cm,yshift=-0.4cm] at (current page.north east) {
        \CategoryBadge[LangColor!20]{Language Feature}
      };
    \end{tikzpicture}
    Answer \theqcounter: ** What are lambda expressions?%
  }

  {\footnotesize
    Lambdas (\texttt{(args) => expr/body}) provide concise syntax for anonymous functions, widely used with LINQ and delegates.
  }

  \begin{minted}{csharp}
Func<int,int> sq = x => x * x;
Console.WriteLine(sq(5)); // 25
  \end{minted}
\end{frame}

% 35 ── Type safety
\QuestionSlide[\CategoryBadge[LangColor!20]{Language Feature}]
  {** What does type-safe mean?}
\begin{frame}[fragile]
  \frametitle{%
    \begin{tikzpicture}[remember picture,overlay]
      \node[anchor=north east,xshift=-0.4cm,yshift=-0.4cm] at (current page.north east) {
        \CategoryBadge[LangColor!20]{Language Feature}
      };
    \end{tikzpicture}
    Answer \theqcounter: ** What does type-safe mean?%
  }

  {\footnotesize
    Type-safe code enforces consistent use of data types at compile time, preventing type errors and reducing runtime exceptions.
  }

  \begin{minted}{csharp}
// Not type-safe
ArrayList list = new ArrayList();
list.Add(42);
int x = (int)list[0]; // cast needed

// Type-safe
List<int> ints = new List<int>();
ints.Add(42);
int y = ints[0];      // no cast
  \end{minted}
\end{frame}

% 36 ── Generics
\QuestionSlide[\CategoryBadge[LangColor!20]{Language Feature}]
  {** What are generics?}
\begin{frame}[fragile]
  \frametitle{%
    \begin{tikzpicture}[remember picture,overlay]
      \node[anchor=north east,xshift=-0.4cm,yshift=-0.4cm] at (current page.north east) {
        \CategoryBadge[LangColor!20]{Language Feature}
      };
    \end{tikzpicture}
    Answer \theqcounter: ** What are generics?%
  }

  {\footnotesize
    Generics allow defining classes and methods with type parameters (e.g., \texttt{List<T>}), enabling type-safe, reusable data structures without performance penalty.
  }

  \begin{minted}{csharp}
public class Box<T> { public T Value { get; set; } }
var intBox = new Box<int> { Value = 42 };
var strBox = new Box<string> { Value = "hello" };
  \end{minted}
\end{frame}

% 37 ── Exception handling
\QuestionSlide[\CategoryBadge[LangColor!20]{Language Feature}]
  {** What is exception handling in C\#?}
\begin{frame}[fragile]
  \frametitle{%
    \begin{tikzpicture}[remember picture,overlay]
      \node[anchor=north east,xshift=-0.4cm,yshift=-0.4cm] at (current page.north east) {
        \CategoryBadge[LangColor!20]{Language Feature}
      };
    \end{tikzpicture}
    Answer \theqcounter: ** What is exception handling in C\#?%
  }

  {\footnotesize
    Exception handling uses \texttt{try}, \texttt{catch}, \texttt{finally}, and \texttt{throw} to manage runtime errors and ensure resource cleanup.
  }

  \begin{minted}{csharp}
try
{
  int r = 10 / divisor;
}
catch (DivideByZeroException)
{
  Console.WriteLine("Cannot divide by zero.");
}
finally
{
  Console.WriteLine("Cleanup.");
}
  \end{minted}
\end{frame}

% 38 ── IDisposable
\QuestionSlide[\CategoryBadge[LangColor!20]{Language Feature}]
  {** What is the \texttt{IDisposable} interface, and why implement it?}
\begin{frame}[fragile]
  \frametitle{%
    \begin{tikzpicture}[remember picture,overlay]
      \node[anchor=north east,xshift=-0.4cm,yshift=-0.4cm] at (current page.north east) {
        \CategoryBadge[LangColor!20]{Language Feature}
      };
    \end{tikzpicture}
    Answer \theqcounter: ** What is the \texttt{IDisposable} interface, and why implement it?%
  }

  {\footnotesize
    \texttt{IDisposable} defines \texttt{Dispose()} to release unmanaged resources (file handles, DB connections). Implementing it allows deterministic cleanup and the \texttt{using} pattern.
  }

  \begin{minted}{csharp}
public class FileLogger : IDisposable
{
  private StreamWriter _w = new("log.txt");
  public void Log(string m) => _w.WriteLine(m);
  public void Dispose() => _w.Dispose();
}

using var logger = new FileLogger();
logger.Log("Start");
  \end{minted}
\end{frame}

% 39 ── using statement
\QuestionSlide[\CategoryBadge[LangColor!20]{Language Feature}]
  {** What is the \texttt{using} statement and \texttt{IDisposable}?}
\begin{frame}[fragile]
  \frametitle{%
    \begin{tikzpicture}[remember picture,overlay]
      \node[anchor=north east,xshift=-0.4cm,yshift=-0.4cm] at (current page.north east) {
        \CategoryBadge[LangColor!20]{Language Feature}
      };
    \end{tikzpicture}
    Answer \theqcounter: ** What is the \texttt{using} statement and \texttt{IDisposable}?%
  }

  {\footnotesize
    The \texttt{using} statement ensures \texttt{Dispose()} is called on \texttt{IDisposable} objects when the block exits, even if an exception occurs.
  }

  \begin{minted}{csharp}
using (var fs = File.OpenRead("data.txt"))
{
  // fs.Dispose() called automatically
}
  \end{minted}
\end{frame}

% 40 ── Tuples
\QuestionSlide[\CategoryBadge[LangColor!20]{Language Feature}]
  {** What are tuples in C\#?}
\begin{frame}[fragile]
  \frametitle{%
    \begin{tikzpicture}[remember picture,overlay]
      \node[anchor=north east,xshift=-0.4cm,yshift=-0.4cm] at (current page.north east) {
        \CategoryBadge[LangColor!20]{Language Feature}
      };
    \end{tikzpicture}
    Answer \theqcounter: ** What are tuples in C\#?%
  }

  {\footnotesize
    Tuples group multiple values into one. Unnamed (\texttt{(int,string)}) or named (\texttt{(int Id,string Name)}), with deconstruction support, useful for returning multiple values without custom types.
  }

  \begin{minted}{csharp}
var user = (Id:1,Name:"Alice");
Console.WriteLine(user.Name); // Alice
var (i,n) = user;
Console.WriteLine($"{i},{n}");
  \end{minted}
\end{frame}

% 41 ── When to use tuples
\QuestionSlide[\CategoryBadge[LangColor!20]{Language Feature}]
  {** When to use tuples?}
\begin{frame}[fragile]
  \frametitle{%
    \begin{tikzpicture}[remember picture,overlay]
      \node[anchor=north east,xshift=-0.4cm,yshift=-0.4cm] at (current page.north east) {
        \CategoryBadge[LangColor!20]{Language Feature}
      };
    \end{tikzpicture}
    Answer \theqcounter: ** When to use tuples?%
  }

  {\footnotesize
    Use tuples for quick, one-off groupings or returning multiple values from methods where defining a class is overkill.
  }

  \begin{minted}{csharp}
(int Min,int Max) GetRange(int[] a)=>
  (a.Min(),a.Max());
var r = GetRange(new[]{3,8,1});
Console.WriteLine($"Min:{r.Min},Max:{r.Max}");
  \end{minted}
\end{frame}

% 42 ── Tuple types
\QuestionSlide[\CategoryBadge[LangColor!20]{Language Feature}]
  {** Tuple is reference or value type?}
\begin{frame}[fragile]
  \frametitle{%
    \begin{tikzpicture}[remember picture,overlay]
      \node[anchor=north east,xshift=-0.4cm,yshift=-0.4cm] at (current page.north east) {
        \CategoryBadge[LangColor!20]{Language Feature}
      };
    \end{tikzpicture}
    Answer \theqcounter: ** Tuple is reference or value type?%
  }

  {\footnotesize
    \texttt{System.Tuple<…>} is a reference type.  
    C\# 7+ syntax \texttt{(T1,…)} maps to \texttt{System.ValueTuple<…>}, a value type with deconstruction support.
  }

  \begin{minted}{csharp}
var refT = Tuple.Create(1,"A");   // reference
var valT = (Id:1,Name:"A");       // value type (ValueTuple)
  \end{minted}
\end{frame}

% 43 ── Value tuples
\QuestionSlide[\CategoryBadge[LangColor!20]{Language Feature}]
  {** What are value tuples?}
\begin{frame}[fragile]
  \frametitle{%
    \begin{tikzpicture}[remember picture,overlay]
      \node[anchor=north east,xshift=-0.4cm,yshift=-0.4cm] at (current page.north east) {
        \CategoryBadge[LangColor!20]{Language Feature}
      };
    \end{tikzpicture}
    Answer \theqcounter: ** What are value tuples?%
  }

  {\footnotesize
    \texttt{ValueTuple<…>} are lightweight, mutable structs with named fields and deconstruction, offering better performance than \texttt{System.Tuple<…>}.
  }

  \begin{minted}{csharp}
var vt = (Id:1,Name:"A");
(int id,string name) = vt;
  \end{minted}
\end{frame}

% 44 ── Nullable types
\QuestionSlide[\CategoryBadge[LangColor!20]{Language Feature}]
  {** What are nullable types?}
\begin{frame}[fragile]
  \frametitle{%
    \begin{tikzpicture}[remember picture,overlay]
      \node[anchor=north east,xshift=-0.4cm,yshift=-0.4cm] at (current page.north east) {
        \CategoryBadge[LangColor!20]{Language Feature}
      };
    \end{tikzpicture}
    Answer \theqcounter: ** What are nullable types?%
  }

  {\footnotesize
    Value types can be made nullable with \texttt{?} (e.g., \texttt{int?}), providing \texttt{HasValue} and \texttt{Value}.
  }

  \begin{minted}{csharp}
int? age = null;
if (age.HasValue) Console.WriteLine(age.Value);
  \end{minted}
\end{frame}

% 45 ── Extension methods
\QuestionSlide[\CategoryBadge[LangColor!20]{Language Feature}]
  {** What are extension methods?}
\begin{frame}[fragile]
  \frametitle{%
    \begin{tikzpicture}[remember picture,overlay]
      \node[anchor=north east,xshift=-0.4cm,yshift=-0.4cm] at (current page.north east) {
        \CategoryBadge[LangColor!20]{Language Feature}
      };
    \end{tikzpicture}
    Answer \theqcounter: ** What are extension methods?%
  }

  {\footnotesize
    Static methods in static classes with \texttt{this} on the first parameter, enabling adding methods to existing types without inheritance.
  }

  \begin{minted}{csharp}
public static class StrExt
{
  public static bool IsCapital(this string s)=>
    !string.IsNullOrEmpty(s)&&char.IsUpper(s[0]);
}
Console.WriteLine("Alice".IsCapital()); // True
  \end{minted}
\end{frame}

% 46 ── Attributes
\QuestionSlide[\CategoryBadge[LangColor!20]{Language Feature}]
  {** What are attributes?}
\begin{frame}[fragile]
  \frametitle{%
    \begin{tikzpicture}[remember picture,overlay]
      \node[anchor=north east,xshift=-0.4cm,yshift=-0.4cm] at (current page.north east) {
        \CategoryBadge[LangColor!20]{Language Feature}
      };
    \end{tikzpicture}
    Answer \theqcounter: ** What are attributes?%
  }

  {\footnotesize
    Attributes are metadata annotations (\texttt{[AttrName]}), used by reflection, serializers, and frameworks to influence behavior at runtime or compile time.
  }

  \begin{minted}{csharp}
[Obsolete("Use NewMethod instead")]
public void OldMethod() { }
  \end{minted}
\end{frame}

% 47 ── Virtual methods
\QuestionSlide[\CategoryBadge[LangColor!20]{Language Feature}]
  {** What are virtual methods in C\#?}
\begin{frame}[fragile]
  \frametitle{%
    \begin{tikzpicture}[remember picture,overlay]
      \node[anchor=north east,xshift=-0.4cm,yshift=-0.4cm] at (current page.north east) {
        \CategoryBadge[LangColor!20]{Language Feature}
      };
    \end{tikzpicture}
    Answer \theqcounter: ** What are virtual methods in C\#?%
  }

  {\footnotesize
    Virtual methods marked with \texttt{virtual} in a base class can be overridden by derived classes. Dispatch is determined by the runtime type via v-table.
  }

  \begin{minted}{csharp}
public class Animal { public virtual void Speak()=>Console.WriteLine("..."); }
public class Cat:Animal{ public override void Speak()=>Console.WriteLine("Meow"); }
Animal a=new Cat(); a.Speak(); // Meow
  \end{minted}
\end{frame}

% 48 ── Virtual properties
\QuestionSlide[\CategoryBadge[LangColor!20]{Language Feature}]
  {** What are virtual properties in C\#?}
\begin{frame}[fragile]
  \frametitle{%
    \begin{tikzpicture}[remember picture,overlay]
      \node[anchor=north east,xshift=-0.4cm,yshift=-0.4cm] at (current page.north east) {
        \CategoryBadge[LangColor!20]{Language Feature}
      };
    \end{tikzpicture}
    Answer \theqcounter: ** What are virtual properties in C\#?%
  }

  {\footnotesize
    Virtual properties allow derived classes to override getters/setters. Mark base with \texttt{virtual} and derived with \texttt{override} for polymorphic behavior.
  }

  \begin{minted}{csharp}
public class Person{ public virtual string Name { get; set; }="X"; }
public class Employee:Person{ public override string Name { get; set; }="Y"; }
Person p=new Employee(); Console.WriteLine(p.Name); // Y
  \end{minted}
\end{frame}

% 49 ── Virtual properties in EF
\QuestionSlide[\CategoryBadge[LangColor!20]{Language Feature}]
  {** Where are virtual properties used in C\#? (Example with EF)}
\begin{frame}[fragile]
  \frametitle{%
    \begin{tikzpicture}[remember picture,overlay]
      \node[anchor=north east,xshift=-0.4cm,yshift=-0.4cm] at (current page.north east) {
        \CategoryBadge[LangColor!20]{Language Feature}
      };
    \end{tikzpicture}
    Answer \theqcounter: ** Where are virtual properties used in C\#? (Example with EF)%
  }

  {\footnotesize
    Entity Framework uses virtual navigation properties to enable lazy loading. EF generates proxies that override these properties to load related entities on demand.
  }

  \begin{minted}{csharp}
public class Order
{
  public int Id { get; set; }
  public virtual Customer Customer { get; set; } // EF proxy overrides for lazy load
}
  \end{minted}
\end{frame}


\hypertarget{sec3}{}
\section{C\# Language: Advanced}
\QuestionSlide[\CategoryBadge[OOPColor!20]{Language \& Platform}]{* What is C\#?}
\begin{frame}[fragile]
  \frametitle{Answer \theqcounter: * What is C\#?}

  \begin{tikzpicture}[remember picture,overlay]
    \node[anchor=north east,xshift=-0.5cm,yshift=-0.5cm]
      at (current page.north east)
      {\CategoryBadge[OOPColor!20]{Language \& Platform}};
  \end{tikzpicture}

  {\footnotesize
    C\# is a modern, object-oriented language developed by Microsoft as part of the .NET ecosystem. It features strong static typing, garbage collection, LINQ for data queries, async/await for asynchronous programming, and interoperability across different platforms.
  }

  \begin{minted}{csharp}
using System;

public class HelloWorld
{
    public static void Main()
    {
        Console.WriteLine("Hello, World!");
    }
}
  \end{minted}
\end{frame}

% 2 ── Span<T>
\QuestionSlide[\CategoryBadge[PerfColor!20]{Performance}]{*** What is \texttt{Span<T>}?}
\begin{frame}[fragile]
  \frametitle{Answer \theqcounter: *** What is \texttt{Span<T>}?}

  \begin{tikzpicture}[remember picture,overlay]
    \node[anchor=north east,xshift=-0.5cm,yshift=-0.5cm]
      at (current page.north east)
      {\CategoryBadge[PerfColor!20]{Performance}};
  \end{tikzpicture}

  {\footnotesize
    \texttt{Span<T>} is a stack-only struct that provides a safe, memory-efficient “window” over contiguous data (arrays, strings, or unmanaged memory) without allocations. It’s ideal for high-performance scenarios like parsing or buffer manipulation because it avoids copying and heap allocations.
  }

  \begin{minted}{csharp}
int[] numbers = { 1, 2, 3, 4, 5 };
Span<int> slice = numbers.AsSpan(1, 3);  // view [2, 3, 4]

slice[0] = 42;                           // writes into numbers[1]
Console.WriteLine(numbers[1]);          // outputs 42
  \end{minted}
\end{frame}

% 3 ── Code weaving
\QuestionSlide[\CategoryBadge[MetaColor!20]{Metaprogramming}]{** What is code weaving in .NET?}
\begin{frame}[fragile]
  \frametitle{Answer \theqcounter: ** What is code weaving in .NET?}

  \begin{tikzpicture}[remember picture,overlay]
    \node[anchor=north east,xshift=-0.5cm,yshift=-0.5cm]
      at (current page.north east)
      {\CategoryBadge[MetaColor!20]{Metaprogramming}};
  \end{tikzpicture}

  {\footnotesize
    Code weaving is a post-compile process where tools (like Fody) inject or modify IL code in assemblies. It’s used for cross-cutting concerns—such as logging, validation, or performance instrumentation—without cluttering source code, though it can make debugging more complex.
  }

  \begin{minted}{csharp}
// Example: Fody injecting INotifyPropertyChanged
public class Person : INotifyPropertyChanged
{
    public string Name { get; set; }  // Fody weaves in the change-notification boilerplate
}
  \end{minted}
\end{frame}

% 4 ── Cross-cutting concerns
\QuestionSlide[\CategoryBadge[ArchColor!20]{Architecture}]{** What are cross-cutting concerns?}
\begin{frame}[fragile]
  \frametitle{Answer \theqcounter: ** What are cross-cutting concerns?}

  \begin{tikzpicture}[remember picture,overlay]
    \node[anchor=north east,xshift=-0.5cm,yshift=-0.5cm]
      at (current page.north east)
      {\CategoryBadge[ArchColor!20]{Architecture}};
  \end{tikzpicture}

  {\footnotesize
    Cross-cutting concerns are aspects that span multiple parts of an application—like logging, security, caching, or error handling. Instead of scattering their code everywhere, you centralize them (via middleware, AOP, or source generators) to keep business logic clean and DRY.
  }

  \begin{minted}{csharp}
// Logging via ASP.NET Core middleware
public class LoggingMiddleware
{
    private readonly RequestDelegate _next;
    public LoggingMiddleware(RequestDelegate next) => _next = next;

    public async Task Invoke(HttpContext context)
    {
        Console.WriteLine($"Request: {context.Request.Path}");
        await _next(context);
        Console.WriteLine($"Response: {context.Response.StatusCode}");
    }
}
  \end{minted}
\end{frame}

% 5 ── Source generators
\QuestionSlide[\CategoryBadge[MetaColor!20]{Metaprogramming}]{*** What are source generators?}
\begin{frame}[fragile]
  \frametitle{Answer \theqcounter: *** What are source generators?}

  \begin{tikzpicture}[remember picture,overlay]
    \node[anchor=north east,xshift=-0.5cm,yshift=-0.5cm]
      at (current page.north east)
      {\CategoryBadge[MetaColor!20]{Metaprogramming}};
  \end{tikzpicture}

  {\footnotesize
    Source generators run at compile time to analyze your code and produce new C\# files. They eliminate reflection and boilerplate by generating optimized code—ideal for scenarios like JSON serializers, dependency injection makers, or performance-critical scaffolding.
  }

  \begin{minted}{csharp}
[Generator]
public class HelloWorldGenerator : ISourceGenerator
{
    public void Initialize(GeneratorInitializationContext context) { }

    public void Execute(GeneratorExecutionContext context)
    {
        context.AddSource("Hello.g.cs", @"
            public static class Hello
            {
                public static void SayHi() =>
                    Console.WriteLine(""Hello from generated code!"");
            }");
    }
}
  \end{minted}
\end{frame}

% 6 ── Expression trees
\QuestionSlide[\CategoryBadge[MetaColor!20]{Metaprogramming}]{*** What are expression trees?}
\begin{frame}[fragile]
  \frametitle{Answer \theqcounter: *** What are expression trees?}

  \begin{tikzpicture}[remember picture,overlay]
    \node[anchor=north east,xshift=-0.5cm,yshift=-0.5cm]
      at (current page.north east)
      {\CategoryBadge[MetaColor!20]{Metaprogramming}};
  \end{tikzpicture}

  {\footnotesize
    Expression trees represent code as a data structure (e.g.\ \texttt{Expression<Func<T,bool>>}). LINQ providers use them to translate C\# lambda expressions into SQL, JSON queries, or other domain-specific languages.
  }

  \begin{minted}{csharp}
using System.Linq.Expressions;

Expression<Func<int, bool>> isEven = x => x % 2 == 0;

// Inspect the tree
Console.WriteLine(isEven.Body); // prints "(x % 2) == 0"
  \end{minted}
\end{frame}


\hypertarget{sec4}{}
\section{LINQ}
% 1 ── What is LINQ?
\QuestionSlide[\CategoryBadge[LinqColor!20]{LINQ}]{** What is LINQ?}
\begin{frame}[fragile]
  \frametitle{%
    \begin{tikzpicture}[remember picture, overlay]
      \node[anchor=north east, xshift=-0.4cm, yshift=-0.4cm, text=black] at (current page.north east) {
        \CategoryBadge[LinqColor!20]{LINQ}
      };
    \end{tikzpicture}
    Answer \theqcounter: What is LINQ?%
  }
  {\footnotesize
  Language Integrated Query enables querying collections with SQL-like syntax or fluent methods (\texttt{Where}, \texttt{Select}, etc.).
  }
\end{frame}

% 2 ── .Select() in LINQ
\QuestionSlide[\CategoryBadge[LinqColor!20]{LINQ}]{* What does \texttt{.Select()} do in LINQ?}
\begin{frame}[fragile]
  \frametitle{%
    \begin{tikzpicture}[remember picture, overlay]
      \node[anchor=north east, xshift=-0.4cm, yshift=-0.4cm, text=black] at (current page.north east) {
        \CategoryBadge[LinqColor!20]{LINQ}
      };
    \end{tikzpicture}
    Answer \theqcounter: What does \texttt{.Select()} do in LINQ?%
  }
  {\footnotesize
  \texttt{.Select()} projects each element of a sequence into a new form. It returns a collection where each element corresponds 1-to-1 with the input sequence.
  }
  \begin{minted}{csharp}
  // Get customer names from WorldWideImporters
  var names = context.Customers
                     .Select(c => c.CustomerName)
                     .ToList();
  \end{minted}
\end{frame}

% 3 ── .SelectMany() in LINQ
\QuestionSlide[\CategoryBadge[LinqColor!20]{LINQ}]{** What does \texttt{.SelectMany()} do in LINQ?}
\begin{frame}[fragile]
  \frametitle{%
    \begin{tikzpicture}[remember picture, overlay]
      \node[anchor=north east, xshift=-0.4cm, yshift=-0.4cm, text=black] at (current page.north east) {
        \CategoryBadge[LinqColor!20]{LINQ}
      };
    \end{tikzpicture}
    Answer \theqcounter: What does \texttt{.SelectMany()} do in LINQ?%
  }
  {\footnotesize
  \texttt{.SelectMany()} flattens nested collections. It projects each element to a collection, then merges all collections into a single flat result set.
  }
  \begin{minted}{csharp}
  // Get all InvoiceLines from all Invoices
  var allLines = context.Invoices
                        .SelectMany(i => i.InvoiceLines)
                        .ToList();
  \end{minted}
\end{frame}

% 4 ── When to use .SelectMany()
\QuestionSlide[\CategoryBadge[LinqColor!20]{LINQ}]{** When would you use \texttt{.SelectMany()} instead of \texttt{.Select()}?}
\begin{frame}[fragile]
  \frametitle{%
    \begin{tikzpicture}[remember picture, overlay]
      \node[anchor=north east, xshift=-0.4cm, yshift=-0.4cm, text=black] at (current page.north east) {
        \CategoryBadge[LinqColor!20]{LINQ}
      };
    \end{tikzpicture}
    Answer \theqcounter: When would you use \texttt{.SelectMany()} instead of \texttt{.Select()}?%
  }
  {\footnotesize
  Use \texttt{.SelectMany()} when projecting nested collections and you want to work with their individual elements directly in a flat structure.
  }
  \begin{minted}{csharp}
  // .Select() keeps nesting
  var nested = context.Customers
                      .Select(c => c.Orders)
                      .ToList(); // List<ICollection<Order>>

  // .SelectMany() flattens to List<Order>
  var flat = context.Customers
                    .SelectMany(c => c.Orders)
                    .ToList();
  \end{minted}
  {\footnotesize
  \texttt{.Select()} → nested structure  
  \texttt{.SelectMany()} → flat structure
  }
\end{frame}


\hypertarget{sec5}{}
\section{Threading \& Async/Await}
% 1 ── What is a thread?
\QuestionSlide[\CategoryBadge[PerfColor!20]{Async \& Threading}]{** What is a thread?}

\begin{frame}[fragile]
  \frametitle{%
    \begin{tikzpicture}[remember picture, overlay]
      \node[anchor=north east, xshift=-0.4cm, yshift=-0.4cm, text=black] at (current page.north east) {
        \CategoryBadge[OOPColor!20]{C\#}
        \CategoryBadge[PerfColor!20]{Async \& Threading}
      };
    \end{tikzpicture}

    Answer \theqcounter: What is a thread?%
  }

  {\footnotesize
  A thread is the basic unit of CPU execution within a process.\\
  Each thread runs code independently and has its own call stack.\\
  Multiple threads can run concurrently on different cores.
  }

  \begin{minted}{csharp}
var t = new Thread(() => Console.WriteLine("Hello from thread"));
t.Start();
t.Join();
  \end{minted}
\end{frame}

% 2 ── Synchronous vs asynchronous code
\QuestionSlide[\CategoryBadge[PerfColor!20]{Async \& Threading}]{** What is the difference between synchronous and asynchronous code?}

\begin{frame}[fragile]
  \frametitle{%
    \begin{tikzpicture}[remember picture, overlay]
      \node[anchor=north east, xshift=-0.4cm, yshift=-0.4cm, text=black] at (current page.north east) {
        \CategoryBadge[OOPColor!20]{C\#}
        \CategoryBadge[PerfColor!20]{Async \& Threading}
      };
    \end{tikzpicture}

    Answer \theqcounter: What is the difference between synchronous and asynchronous code?%
  }

  {\footnotesize
  Synchronous code blocks the calling thread until work completes.\\
  Asynchronous code lets the thread continue while the operation finishes later.
  }

  \begin{minted}{csharp}
Thread.Sleep(1000);           // synchronous pause
await Task.Delay(1000);       // asynchronous pause
  \end{minted}
\end{frame}

% 3 ── Task basics
\QuestionSlide[\CategoryBadge[PerfColor!20]{Async \& Threading}]{** What is a \texttt{Task}?}

\begin{frame}[fragile]
  \frametitle{%
    \begin{tikzpicture}[remember picture, overlay]
      \node[anchor=north east, xshift=-0.4cm, yshift=-0.4cm, text=black] at (current page.north east) {
        \CategoryBadge[OOPColor!20]{C\#}
        \CategoryBadge[PerfColor!20]{Async \& Threading}
      };
    \end{tikzpicture}

    Answer \theqcounter: What is a \texttt{Task}?%
  }

  {\footnotesize
  A \texttt{Task} represents an asynchronous operation that may produce a result.\\
  It can run on the thread pool and can be awaited or combined with other tasks.
  }

  \begin{minted}{csharp}
Task<int> work = Task.Run(() => 42);
int result = await work;
  \end{minted}
\end{frame}

% 4 ── The \texttt{async} keyword
\QuestionSlide[\CategoryBadge[PerfColor!20]{Async \& Threading}]{** What does the \texttt{async} keyword do?}

\begin{frame}[fragile]
  \frametitle{%
    \begin{tikzpicture}[remember picture, overlay]
      \node[anchor=north east, xshift=-0.4cm, yshift=-0.4cm, text=black] at (current page.north east) {
        \CategoryBadge[OOPColor!20]{C\#}
        \CategoryBadge[PerfColor!20]{Async \& Threading}
      };
    \end{tikzpicture}

    Answer \theqcounter: What does the \texttt{async} keyword do?%
  }

  {\footnotesize
  The \texttt{async} keyword marks a method as asynchronous and enables the use of \texttt{await}.\\
  It causes the compiler to transform the method into a state machine that returns a \texttt{Task}.
  }

  \begin{minted}{csharp}
public async Task<int> GetValueAsync()
{
    await Task.Delay(100);
    return 42;
}
  \end{minted}
\end{frame}

% 5 ── The \texttt{await} keyword
\QuestionSlide[\CategoryBadge[PerfColor!20]{Async \& Threading}]{** What does the \texttt{await} keyword do?}

\begin{frame}[fragile]
  \frametitle{%
    \begin{tikzpicture}[remember picture, overlay]
      \node[anchor=north east, xshift=-0.4cm, yshift=-0.4cm, text=black] at (current page.north east) {
        \CategoryBadge[OOPColor!20]{C\#}
        \CategoryBadge[PerfColor!20]{Async \& Threading}
      };
    \end{tikzpicture}

    Answer \theqcounter: What does the \texttt{await} keyword do?%
  }

  {\footnotesize
  \texttt{await} asynchronously pauses the method until the awaited task completes.\\
  It frees the calling thread and resumes the method's continuation later.
  }

  \begin{minted}{csharp}
await Task.Delay(1000);
Console.WriteLine("Done");
  \end{minted}
\end{frame}

% 6 ── Forgetting to await
\QuestionSlide[\CategoryBadge[PerfColor!20]{Async \& Threading}]{** What happens if you forget to \texttt{await} an async method?}

\begin{frame}[fragile]
  \frametitle{%
    \begin{tikzpicture}[remember picture, overlay]
      \node[anchor=north east, xshift=-0.4cm, yshift=-0.4cm, text=black] at (current page.north east) {
        \CategoryBadge[OOPColor!20]{C\#}
        \CategoryBadge[PerfColor!20]{Async \& Threading}
      };
    \end{tikzpicture}

    Answer \theqcounter: What happens if you forget to \texttt{await} an async method?%
  }

  {\footnotesize
  The method starts but continues in the background.\\
  You won't catch exceptions or know when it finishes.\\
  In console apps, the process may exit before the task completes.
  }

  \begin{minted}{csharp}
public async Task LogAsync() =>
    await File.AppendAllTextAsync("log.txt", "Event\n");

public void Main()
{
    LogAsync(); // not awaited
    Console.WriteLine("Done"); // may run before write completes
}
  \end{minted}
\end{frame}

% 7 ── Does async always make code faster?
\QuestionSlide[\CategoryBadge[PerfColor!20]{Async \& Threading}]{** Does async always make code faster?}

\begin{frame}[fragile]
  \frametitle{%
    \begin{tikzpicture}[remember picture, overlay]
      \node[anchor=north east, xshift=-0.4cm, yshift=-0.4cm, text=black] at (current page.north east) {
        \CategoryBadge[OOPColor!20]{C\#}
        \CategoryBadge[PerfColor!20]{Async \& Threading}
      };
    \end{tikzpicture}

    Answer \theqcounter: Does async always make code faster?%
  }

  {\footnotesize
  \begin{itemize}
      \item Async improves scalability for I/O-bound tasks but \textbf{doesn't} make CPU work faster.
      \item Async introduces overhead; it's not a free performance boost.
  \end{itemize}
  }

  \begin{minted}{csharp}
public async Task FetchDataAsync()
{
    var client = new HttpClient();
    string data = await client.GetStringAsync("https://api.example.com");
    Console.WriteLine(data.Length);
}
  \end{minted}
\end{frame}

% 8 ── Why use async in different app types?
\QuestionSlide[\CategoryBadge[PerfColor!20]{Async \& Threading}]{** Why use async in console, desktop, and web apps?}

\begin{frame}[fragile]
  \frametitle{%
    \begin{tikzpicture}[remember picture, overlay]
      \node[anchor=north east, xshift=-0.4cm, yshift=-0.4cm, text=black] at (current page.north east) {
        \CategoryBadge[OOPColor!20]{C\#}
        \CategoryBadge[PerfColor!20]{Async \& Threading}
      };
    \end{tikzpicture}

    Answer \theqcounter: Why use async in console, desktop, and web apps?%
  }

  {\footnotesize
  \textbf{Console}: avoid blocking on I/O\\
  \textbf{Desktop}: keep UI responsive\\
  \textbf{Web}: improve request throughput by freeing server threads during I/O waits
  }

  \begin{minted}{csharp}
// Console app
public static async Task Main()
{
    var html = await new HttpClient()
        .GetStringAsync("https://example.com");

    Console.WriteLine(html[..80]);
}
  \end{minted}
\end{frame}

% 9 ── Async vs multithreading
\QuestionSlide[\CategoryBadge[PerfColor!20]{Async \& Threading}]{** What's the difference between async and multithreading?}

\begin{frame}[fragile]
  \frametitle{%
    \begin{tikzpicture}[remember picture, overlay]
      \node[anchor=north east, xshift=-0.4cm, yshift=-0.4cm, text=black] at (current page.north east) {
        \CategoryBadge[OOPColor!20]{C\#}
        \CategoryBadge[PerfColor!20]{Async \& Threading}
      };
    \end{tikzpicture}

    Answer \theqcounter: What's the difference between async and multithreading?%
  }

  {\footnotesize
  Async is non-blocking and usually single-threaded, using an event loop.\\
  Multithreading runs work in parallel on multiple threads.\\
  Use async for I/O-bound tasks, multithreading for CPU-bound work.
  }

  \begin{minted}{csharp}
// Asynchronous (non-blocking)
public async Task<string> LoadFileAsync() =>
    await File.ReadAllTextAsync("log.txt");

// Multithreading (CPU-bound)
public Task<int> HeavyWork() =>
    Task.Run(() => Enumerable.Range(1, 100000).Sum());
  \end{minted}
\end{frame}

% 10 ── Thread-pool basics
\QuestionSlide[\CategoryBadge[PerfColor!20]{Async \& Threading}]{** What is the .NET Thread Pool?}

\begin{frame}[fragile]
  \frametitle{%
    \begin{tikzpicture}[remember picture, overlay]
      \node[anchor=north east, xshift=-0.4cm, yshift=-0.4cm, text=black] at (current page.north east) {
        \CategoryBadge[OOPColor!20]{C\#}
        \CategoryBadge[PerfColor!20]{Async \& Threading}
      };
    \end{tikzpicture}

    Answer \theqcounter: What is the .NET Thread Pool?%
  }

  {\footnotesize
  The thread pool is a shared set of worker threads managed by the CLR.\\
  It reuses threads for short-lived or asynchronous work, avoiding the cost of creating new threads.
  }

  \begin{minted}{csharp}
Task.Run(() =>
    Console.WriteLine($"Pool thread: {Thread.CurrentThread.IsThreadPoolThread}"));
  \end{minted}
\end{frame}

% 11 ── Task.Run vs Thread
\QuestionSlide[\CategoryBadge[PerfColor!20]{Async \& Threading}]{** What's the difference between \texttt{Task.Run} and \texttt{Thread}?}

\begin{frame}[fragile]
  \frametitle{%
    \begin{tikzpicture}[remember picture, overlay]
      \node[anchor=north east, xshift=-0.4cm, yshift=-0.4cm, text=black] at (current page.north east) {
        \CategoryBadge[OOPColor!20]{C\#}
        \CategoryBadge[PerfColor!20]{Async \& Threading}
      };
    \end{tikzpicture}

    Answer \theqcounter: What's the difference between \texttt{Task.Run} and \texttt{Thread}?%
  }

  {\footnotesize
  \texttt{Task.Run} queues work to the thread pool and returns a \texttt{Task}.\\
  \texttt{new Thread} creates a dedicated OS thread for long-running or special scenarios.\\
  Prefer \texttt{Task.Run} for short CPU-bound work; use \texttt{Thread} when you need explicit thread control.
  }

  \begin{minted}{csharp}
void Work() => Console.WriteLine(Thread.CurrentThread.IsThreadPoolThread);

Task.Run(Work);

var t = new Thread(Work) { IsBackground = true };
t.Start();
  \end{minted}
\end{frame}

% 12 ── When to use Task.Run
\QuestionSlide[\CategoryBadge[PerfColor!20]{Async \& Threading}]{** When should you use \texttt{Task.Run}?}

\begin{frame}[fragile]
  \frametitle{%
    \begin{tikzpicture}[remember picture, overlay]
      \node[anchor=north east, xshift=-0.4cm, yshift=-0.4cm, text=black] at (current page.north east) {
        \CategoryBadge[OOPColor!20]{C\#}
        \CategoryBadge[PerfColor!20]{Async \& Threading}
      };
    \end{tikzpicture}

    Answer \theqcounter: When should you use \texttt{Task.Run}?%
  }

  {\footnotesize
  Use \texttt{Task.Run} to offload CPU-bound work to a background thread.\\
  Don't wrap naturally asynchronous I/O operations with \texttt{Task.Run}.
  }

  \begin{minted}{csharp}
// CPU-bound work offloaded to thread pool
int result = await Task.Run(() =>
    Enumerable.Range(1, 1_000_000).Sum());
  \end{minted}
\end{frame}

% 13 ── Sync vs async endpoint
\QuestionSlide[\CategoryBadge[PerfColor!20]{Async \& Threading}]{** Which is faster: a synchronous or asynchronous endpoint? Why?}

\begin{frame}[fragile]
  \frametitle{%
    \begin{tikzpicture}[remember picture, overlay]
      \node[anchor=north east, xshift=-0.4cm, yshift=-0.4cm, text=black] at (current page.north east) {
        \CategoryBadge[OOPColor!20]{C\#}
        \CategoryBadge[PerfColor!20]{Async \& Threading}
      };
    \end{tikzpicture}

    Answer \theqcounter: Which is faster: a synchronous or asynchronous endpoint? Why?%
  }

  {\footnotesize
  An asynchronous endpoint scales better under load because it doesn't block threads during I/O.\\
  A synchronous version may be slightly faster for a single request but handles fewer concurrent users.
  }

  \begin{minted}{csharp}
// Synchronous
public string GetData() => File.ReadAllText("data.txt");

// Asynchronous
public async Task<string> GetDataAsync() =>
    await File.ReadAllTextAsync("data.txt");
  \end{minted}
\end{frame}

% 14 ── Deadlocks from sync-over-async
\QuestionSlide[\CategoryBadge[PerfColor!20]{Async \& Threading}]{*** How can synchronous code calling asynchronous methods cause deadlocks?}

\begin{frame}[fragile]
  \frametitle{%
    \begin{tikzpicture}[remember picture, overlay]
      \node[anchor=north east, xshift=-0.4cm, yshift=-0.4cm, text=black] at (current page.north east) {
        \CategoryBadge[OOPColor!20]{C\#}
        \CategoryBadge[PerfColor!20]{Async \& Threading}
      };
    \end{tikzpicture}

    Answer \theqcounter: How can synchronous code calling asynchronous methods cause deadlocks?%
  }

  {\footnotesize
  Calling an async method with \texttt{.Result} or \texttt{.Wait()} blocks the thread.\\
  If the continuation tries to run on that same context, a deadlock occurs.
  }

  \begin{minted}{csharp}
// Deadlock example
public async Task<int> LoadAsync()
{
    await Task.Delay(1000);
    return 42;
}

public void Main()
{
    var result = LoadAsync().Result; // Possible deadlock!
}
  \end{minted}
\end{frame}

% 15 ── Exception flow in async methods
\QuestionSlide[\CategoryBadge[PerfColor!20]{Async \& Threading}]{*** How do exceptions flow in async methods?}

\begin{frame}[fragile]
  \frametitle{%
    \begin{tikzpicture}[remember picture, overlay]
      \node[anchor=north east, xshift=-0.4cm, yshift=-0.4cm, text=black] at (current page.north east) {
        \CategoryBadge[OOPColor!20]{C\#}
        \CategoryBadge[PerfColor!20]{Async \& Threading}
      };
    \end{tikzpicture}

    Answer \theqcounter: How do exceptions flow in async methods?%
  }

  {\footnotesize
  Exceptions thrown in an async method are captured in its \texttt{Task}.\\
  They surface when awaited; unawaited tasks may hide errors.\\
  Use \texttt{try}/\texttt{catch} around \texttt{await}; \texttt{Task.WhenAll} aggregates multiple exceptions.
  }

  \begin{minted}{csharp}
public async Task<int> FailAsync()
{
    await Task.Delay(100);
    throw new InvalidOperationException();
}

public async Task DemoAsync()
{
    try
    {
        int n = await FailAsync();
    }
    catch (InvalidOperationException ex)
    {
        Console.WriteLine(ex.Message);
    }
}
  \end{minted}
\end{frame}

% 16 ── Async void pitfalls
\QuestionSlide[\CategoryBadge[PerfColor!20]{Async \& Threading}]{*** Why is \texttt{async void} usually a bad idea?}

\begin{frame}[fragile]
  \frametitle{%
    \begin{tikzpicture}[remember picture, overlay]
      \node[anchor=north east, xshift=-0.4cm, yshift=-0.4cm, text=black] at (current page.north east) {
        \CategoryBadge[OOPColor!20]{C\#}
        \CategoryBadge[PerfColor!20]{Async \& Threading}
      };
    \end{tikzpicture}

    Answer \theqcounter: Why is \texttt{async void} usually a bad idea?%
  }

  {\footnotesize
  \texttt{async void} methods can't be awaited and exceptions bypass callers.\\
  Use them only for event handlers; prefer \texttt{async Task} so callers can await and handle errors.
  }

  \begin{minted}{csharp}
public async void OnClick(object sender, EventArgs e)
{
    await Task.Delay(1000);
    throw new Exception("boom"); // crashes process
}

public async Task OnClickAsync()
{
    await Task.Delay(1000);
}
  \end{minted}
\end{frame}

% 17 ── ConfigureAwait(false)
\QuestionSlide[\CategoryBadge[PerfColor!20]{Async \& Threading}]{*** When and why should you use \texttt{ConfigureAwait(false)} in async methods?}

\begin{frame}[fragile]
  \frametitle{%
    \begin{tikzpicture}[remember picture, overlay]
      \node[anchor=north east, xshift=-0.4cm, yshift=-0.4cm, text=black] at (current page.north east) {
        \CategoryBadge[OOPColor!20]{C\#}
        \CategoryBadge[PerfColor!20]{Async \& Threading}
      };
    \end{tikzpicture}

    Answer \theqcounter: When and why should you use \texttt{ConfigureAwait(false)} in async methods?%
  }

  {\footnotesize
  Use \texttt{ConfigureAwait(false)} in library code to avoid capturing the synchronization context.\\
  It reduces overhead and prevents deadlocks in UI or ASP.NET scenarios.
  }

  \begin{minted}{csharp}
public async Task<string> GetContentAsync()
{
    var client = new HttpClient();
    return await client.GetStringAsync("https://api.com")
                       .ConfigureAwait(false);
}
  \end{minted}
\end{frame}

% 18 ── Parallel.ForEach vs Task.WhenAll
\QuestionSlide[\CategoryBadge[PerfColor!20]{Async \& Threading}]{*** What's the difference between \texttt{Parallel.ForEach()} and \texttt{Task.WhenAll()}?}

\begin{frame}[fragile]
  \frametitle{%
    \begin{tikzpicture}[remember picture, overlay]
      \node[anchor=north east, xshift=-0.4cm, yshift=-0.4cm, text=black] at (current page.north east) {
        \CategoryBadge[OOPColor!20]{C\#}
        \CategoryBadge[PerfColor!20]{Async \& Threading}
      };
    \end{tikzpicture}

    Answer \theqcounter: What's the difference between \texttt{Parallel.ForEach()} and \texttt{Task.WhenAll()}?%
  }

  {\footnotesize
  \textbf{Parallel.ForEach}: for CPU-bound work, uses multiple threads in parallel.\\
  \textbf{Task.WhenAll}: for async I/O, awaits many non-blocking operations concurrently.
  }

  \begin{minted}{csharp}
// CPU-bound (parallel)
Parallel.ForEach(items, item => HeavyCalculation(item));

// I/O-bound (async)
var tasks = urls.Select(u => client.GetAsync(u));
await Task.WhenAll(tasks);
  \end{minted}
\end{frame}

% 19 ── Cancellation tokens
\QuestionSlide[\CategoryBadge[PerfColor!20]{Async \& Threading}]{*** How do you correctly implement cancellation in asynchronous methods?}

\begin{frame}[fragile]
  \frametitle{%
    \begin{tikzpicture}[remember picture, overlay]
      \node[anchor=north east, xshift=-0.4cm, yshift=-0.4cm, text=black] at (current page.north east) {
        \CategoryBadge[OOPColor!20]{C\#}
        \CategoryBadge[PerfColor!20]{Async \& Threading}
      };
    \end{tikzpicture}

    Answer \theqcounter: How do you correctly implement cancellation in asynchronous methods?%
  }

  {\footnotesize
  Accept a \texttt{CancellationToken} and pass it to cancellable operations.\\
  Check \texttt{token.ThrowIfCancellationRequested()} to abort work cooperatively.
  }

  \begin{minted}{csharp}
public async Task DownloadAsync(string url, CancellationToken token)
{
    using var client = new HttpClient();
    var response = await client.GetAsync(url, token);
    response.EnsureSuccessStatusCode();

    token.ThrowIfCancellationRequested();
    var content = await response.Content.ReadAsStringAsync(token);
}
  \end{minted}
\end{frame}

% 20 ── Async streams
\QuestionSlide[\CategoryBadge[PerfColor!20]{Async \& Threading}]{*** What are async streams (\texttt{IAsyncEnumerable<T>})?}

\begin{frame}[fragile]
  \frametitle{%
    \begin{tikzpicture}[remember picture, overlay]
      \node[anchor=north east, xshift=-0.4cm, yshift=-0.4cm, text=black] at (current page.north east) {
        \CategoryBadge[OOPColor!20]{C\#}
        \CategoryBadge[PerfColor!20]{Async \& Threading}
      };
    \end{tikzpicture}

    Answer \theqcounter: What are async streams (\texttt{IAsyncEnumerable<T>})?%
  }

  {\footnotesize
  Async streams let you produce or consume data asynchronously using \texttt{await foreach}.\\
  They combine \texttt{async/await} with \texttt{yield return} to handle data arriving over time.
  }

  \begin{minted}{csharp}
public async IAsyncEnumerable<int> GetNumbersAsync()
{
    for (int i = 1; i <= 3; i++)
    {
        await Task.Delay(500);
        yield return i;
    }
}

public async Task ConsumeAsync()
{
    await foreach (var n in GetNumbersAsync())
        Console.WriteLine(n);
}
  \end{minted}
\end{frame}

% 21 ── IAsyncEnumerable<T> vs Task<IEnumerable<T>>
\QuestionSlide[\CategoryBadge[PerfColor!20]{Async \& Threading}]{*** What's the difference between returning \texttt{IAsyncEnumerable<T>} and \texttt{Task<IEnumerable<T>>}?}

\begin{frame}[fragile]
  \frametitle{%
    \begin{tikzpicture}[remember picture, overlay]
      \node[anchor=north east, xshift=-0.4cm, yshift=-0.4cm, text=black] at (current page.north east) {
        \CategoryBadge[OOPColor!20]{C\#}
        \CategoryBadge[PerfColor!20]{Async \& Threading}
      };
    \end{tikzpicture}

    Answer \theqcounter: What's the difference between returning \texttt{IAsyncEnumerable<T>} and \texttt{Task<IEnumerable<T>>}?%
  }

  {\footnotesize
  \texttt{Task<IEnumerable<T>>} returns the whole collection after completion.\\
  \texttt{IAsyncEnumerable<T>} streams items one by one as they become available.
  }

  \begin{minted}{csharp}
// Fully buffered
public async Task<List<int>> GetAllAsync() =>
    await db.Items.Select(i => i.Id).ToListAsync();

// Streaming
public async IAsyncEnumerable<int> GetStreamAsync()
{
    await foreach (var item in db.Items.AsAsyncEnumerable())
        yield return item.Id;
}
  \end{minted}
\end{frame}

% 22 ── Task vs ValueTask
\QuestionSlide[\CategoryBadge[PerfColor!20]{Async \& Threading}]{*** What's the difference between \texttt{Task} and \texttt{ValueTask}?}

\begin{frame}[fragile]
  \frametitle{%
    \begin{tikzpicture}[remember picture, overlay]
      \node[anchor=north east, xshift=-0.4cm, yshift=-0.4cm, text=black] at (current page.north east) {
        \CategoryBadge[OOPColor!20]{C\#}
        \CategoryBadge[PerfColor!20]{Async \& Threading}
      };
    \end{tikzpicture}

    Answer \theqcounter: What's the difference between \texttt{Task} and \texttt{ValueTask}?%
  }

  {\footnotesize
  \texttt{Task} always allocates an object; \texttt{ValueTask} is a struct that can avoid allocation when a result is already available.\\
  \texttt{ValueTask} should be awaited only once and used in performance-critical paths.
  }

  \begin{minted}{csharp}
public ValueTask<int> GetCachedValueAsync(bool cached)
{
    return cached ? new ValueTask<int>(42)
                  : new ValueTask<int>(LoadAsync());
}
  \end{minted}
\end{frame}

% 23 ── SynchronizationContext
\QuestionSlide[\CategoryBadge[PerfColor!20]{Async \& Threading}]{*** What is the \texttt{SynchronizationContext} and why does it matter?}

\begin{frame}[fragile]
  \frametitle{%
    \begin{tikzpicture}[remember picture, overlay]
      \node[anchor=north east, xshift=-0.4cm, yshift=-0.4cm, text=black] at (current page.north east) {
        \CategoryBadge[OOPColor!20]{C\#}
        \CategoryBadge[PerfColor!20]{Async \& Threading}
      };
    \end{tikzpicture}

    Answer \theqcounter: What is the \texttt{SynchronizationContext} and why does it matter?%
  }

  {\footnotesize
  \texttt{SynchronizationContext} represents the environment that handles continuation scheduling (e.g.\ UI thread).\\
  Capturing it ensures UI updates happen on the correct thread; avoiding it with \texttt{ConfigureAwait(false)} can prevent deadlocks.
  }

  \begin{minted}{csharp}
var ctx = SynchronizationContext.Current;
ctx?.Post(_ => Console.WriteLine("Back on context"), null);
  \end{minted}
\end{frame}


\hypertarget{sec6}{}
\section{Entity Framework}
% 1 ── What is lazy loading in EF
\QuestionSlide[\CategoryBadge[PerfColor!20]{Async \& Threading}]{** What is \textbf{lazy loading} in Entity Framework and how can you disable it?}

\begin{frame}[fragile]
  \frametitle{%
    \begin{tikzpicture}[remember picture, overlay]
      \node[anchor=north east, xshift=-0.4cm, yshift=-0.4cm, text=black] at (current page.north east) {
        \CategoryBadge[OOPColor!20]{C\#}
        \CategoryBadge[PerfColor!20]{Async \& Threading}
      };
    \end{tikzpicture}

    Answer \theqcounter: What is \textbf{lazy loading} in Entity Framework and how can you disable it?%
  }

  {\footnotesize
  Navigation property loads related data on first access via proxy.
  \begin{itemize}
    \item Disable by not using proxy classes
    \item Turning off \texttt{ChangeTracker.LazyLoadingEnabled}
    \item Using EF Core without \texttt{UseLazyLoadingProxies}
    \item Prevents N\,+\,1 by switching to eager loading with \texttt{Include()}
  \end{itemize}
  }

  \begin{minted}{csharp}
  // Lazy loading via proxy (EF Core with UseLazyLoadingProxies)
  public class Blog
  {
      public int Id { get; set; }
      public virtual ICollection<Post> Posts { get; set; }
  }

  // Disabling lazy loading
  context.ChangeTracker.LazyLoadingEnabled = false;

  // Eager loading instead
  var blog = context.Blogs
                    .Include(b => b.Posts)
                    .FirstOrDefault();
  \end{minted}
\end{frame}

% 2 ── Risk of lazy loading
\QuestionSlide[\CategoryBadge[PerfColor!20]{Async \& Threading}]{*** What main risk does lazy loading introduce?}

\begin{frame}[fragile]
  \frametitle{%
    \begin{tikzpicture}[remember picture, overlay]
      \node[anchor=north east, xshift=-0.4cm, yshift=-0.4cm, text=black] at (current page.north east) {
        \CategoryBadge[OOPColor!20]{C\#}
        \CategoryBadge[PerfColor!20]{Async \& Threading}
      };
    \end{tikzpicture}

    Answer \theqcounter: What main risk does lazy loading introduce?%
  }

  {\footnotesize
  It can generate the \textbf{N+1} problem—one query for the root set and
  an additional query for each related entity, causing many round-trips
  and performance degradation.
  }
\end{frame}

% 3 ── Enabling eager loading
\QuestionSlide[\CategoryBadge[PerfColor!20]{Async \& Threading}]{* How do you enable eager loading?}

\begin{frame}[fragile]
  \frametitle{%
    \begin{tikzpicture}[remember picture, overlay]
      \node[anchor=north east, xshift=-0.4cm, yshift=-0.4cm, text=black] at (current page.north east) {
        \CategoryBadge[OOPColor!20]{C\#}
        \CategoryBadge[PerfColor!20]{Async \& Threading}
      };
    \end{tikzpicture}

    Answer \theqcounter: How do you enable eager loading?%
  }

  {\footnotesize
  By chaining \texttt{Include()} / \texttt{ThenInclude()} methods in LINQ.
  EF Core joins or splits queries so that related data are loaded upfront
  in as few SQL commands as possible.
  }
\end{frame}

% 4 ── Advantage of eager loading
\QuestionSlide[\CategoryBadge[PerfColor!20]{Async \& Threading}]{*** Give one advantage of eager loading over lazy loading.}

\begin{frame}[fragile]
  \frametitle{%
    \begin{tikzpicture}[remember picture, overlay]
      \node[anchor=north east, xshift=-0.4cm, yshift=-0.4cm, text=black] at (current page.north east) {
        \CategoryBadge[OOPColor!20]{C\#}
        \CategoryBadge[PerfColor!20]{Async \& Threading}
      };
    \end{tikzpicture}

    Answer \theqcounter: Give one advantage of eager loading over lazy loading.%
  }

  {\footnotesize
  Eager loading eliminates unnecessary round-trips, preventing the N+1 issue
  and reducing total query latency when related data are certainly needed.
  }
\end{frame}

% 5 ── When to prefer lazy loading
\QuestionSlide[\CategoryBadge[PerfColor!20]{Async \& Threading}]{* When might lazy loading be preferred?}

\begin{frame}[fragile]
  \frametitle{%
    \begin{tikzpicture}[remember picture, overlay]
      \node[anchor=north east, xshift=-0.4cm, yshift=-0.4cm, text=black] at (current page.north east) {
        \CategoryBadge[OOPColor!20]{C\#}
        \CategoryBadge[PerfColor!20]{Async \& Threading}
      };
    \end{tikzpicture}

    Answer \theqcounter: When might lazy loading be preferred?%
  }

  {\footnotesize
  In highly interactive UIs or prototypes where most related entities
  are \emph{rarely} accessed, lazy loading avoids fetching large graphs
  that the user never expands.
  }
\end{frame}

% 6 ── SplitQuery vs SingleQuery
\QuestionSlide[\CategoryBadge[PerfColor!20]{Async \& Threading}]{*** What is \texttt{SplitQuery} vs \texttt{SingleQuery} in eager loading?}

\begin{frame}[fragile]
  \frametitle{%
    \begin{tikzpicture}[remember picture, overlay]
      \node[anchor=north east, xshift=-0.4cm, yshift=-0.4cm, text=black] at (current page.north east) {
        \CategoryBadge[OOPColor!20]{C\#}
        \CategoryBadge[PerfColor!20]{Async \& Threading}
      };
    \end{tikzpicture}

    Answer \theqcounter: What is \texttt{SplitQuery} vs \texttt{SingleQuery} in eager loading?%
  }

  {\footnotesize
  \begin{itemize}
    \item \textbf{SingleQuery} loads everything with one SQL statement (using JOINs).
    \item \textbf{SplitQuery} breaks the load into multiple queries—one per collection—to avoid Cartesian explosion.
    \item Choose via \texttt{AsSingleQuery()} or \texttt{AsSplitQuery()}.
  \end{itemize}
  }

  \begin{minted}{csharp}
  // SingleQuery: one big SQL with JOINs
  var blog1 = context.Blogs
      .Include(b => b.Posts)
      .AsSingleQuery()
      .FirstOrDefault();

  // SplitQuery: separate SQL queries per navigation
  var blog2 = context.Blogs
      .Include(b => b.Posts) // first navigation
      .AsSplitQuery()
      .FirstOrDefault();
  \end{minted}

  {\tiny
  per navigation - it means that for each Include(), EF Core will execute one SQL query per relationship rather than combining all relationships into one big query with joins.
  }
\end{frame}

% 7 ── TemporalAsOf usage
\QuestionSlide[\CategoryBadge[PerfColor!20]{Async \& Threading}]{*** When might \texttt{TemporalAsOf} be preferred?}

\begin{frame}[fragile]
  \frametitle{%
    \begin{tikzpicture}[remember picture, overlay]
      \node[anchor=north east, xshift=-0.4cm, yshift=-0.4cm, text=black] at (current page.north east) {
        \CategoryBadge[OOPColor!20]{C\#}
        \CategoryBadge[PerfColor!20]{Async \& Threading}
      };
    \end{tikzpicture}

    Answer \theqcounter: When might \texttt{TemporalAsOf} be preferred?%
  }

  {\footnotesize
  When analysing how data \emph{looked at a specific point in time},
  \texttt{TemporalAsOf} lets you query historical snapshots—for auditing,
  reporting, or reproducing past business logic.

  \begin{itemize}
    \item Past prices, statuses or other time-sensitive data  
    \item Debugging data inconsistencies across time  
    \item Building reports that must reflect a precise “as-of” date
  \end{itemize}
  }

  \begin{minted}{csharp}
  // Historical view as of Jan 1 2013
  context.StockItems
         .TemporalAsOf(new DateTime(2013, 1, 1))
         .Select(si => new { si.StockItemName, si.UnitPrice });
  \end{minted}
\end{frame}


\hypertarget{sec7}{}
\section{Design Principles}
% 1 ── Core OOP paradigms
\QuestionSlide[\CategoryBadge[OOPColor!20]{OOP}]{** What are the core object-oriented programming (OOP) paradigms?}

\begin{frame}[fragile]
  \frametitle{%
    \begin{tikzpicture}[remember picture, overlay]
      \node[anchor=north east, xshift=-0.4cm, yshift=-0.4cm, text=black] at (current page.north east) {
        \CategoryBadge[OOPColor!20]{OOP}
      };
    \end{tikzpicture}

    Answer \theqcounter: What are the core object-oriented programming (OOP) paradigms?%
  }

  {\footnotesize
  Object-Oriented Programming (OOP) is a paradigm centered around objects that encapsulate data and behavior.  
  The four core principles are:
  \begin{itemize}
      \item \textbf{Encapsulation} – Hiding internal state and exposing behavior through methods.
      \item \textbf{Abstraction} – Hiding complex implementation behind a simpler interface.
      \item \textbf{Inheritance} – Deriving new classes from existing ones to reuse code.
      \item \textbf{Polymorphism} – Treating objects of different types uniformly via shared interfaces or base classes.
  \end{itemize}
  }

  \begin{minted}{csharp}
public abstract class Animal // Abstraction
{
    public string Name { get; set; } // Encapsulation
    public abstract void Speak();    // Polymorphism via override
}

public class Dog : Animal
{
    public override void Speak() => Console.WriteLine("Bark");
}
  \end{minted}
\end{frame}

% 2 ── Inheritance in C#
\QuestionSlide[\CategoryBadge[OOPColor!20]{OOP}]{** What is inheritance in C\#?}

\begin{frame}[fragile]
  \frametitle{%
    \begin{tikzpicture}[remember picture, overlay]
      \node[anchor=north east, xshift=-0.4cm, yshift=-0.4cm, text=black] at (current page.north east) {
        \CategoryBadge[OOPColor!20]{OOP}
      };
    \end{tikzpicture}

    Answer \theqcounter: What is inheritance in C\#?%
  }

  {\footnotesize
  Inheritance allows a class (derived) to inherit members from another class (base), enabling code reuse and polymorphism.
  }
\end{frame}

% 3 ── Polymorphism
\QuestionSlide[\CategoryBadge[OOPColor!20]{OOP}]{** What is polymorphism?}

\begin{frame}[fragile]
  \frametitle{%
    \begin{tikzpicture}[remember picture, overlay]
      \node[anchor=north east, xshift=-0.4cm, yshift=-0.4cm, text=black] at (current page.north east) {
        \CategoryBadge[OOPColor!20]{OOP}
      };
    \end{tikzpicture}

    Answer \theqcounter: What is polymorphism?%
  }

  {\footnotesize
  Polymorphism lets objects be treated as instances of their base type, with overriding and virtual methods enabling dynamic behavior.
  }
\end{frame}

% 4 ── Encapsulation
\QuestionSlide[\CategoryBadge[OOPColor!20]{OOP}]{** What is encapsulation?}

\begin{frame}[fragile]
  \frametitle{%
    \begin{tikzpicture}[remember picture, overlay]
      \node[anchor=north east, xshift=-0.4cm, yshift=-0.4cm, text=black] at (current page.north east) {
        \CategoryBadge[OOPColor!20]{OOP}
      };
    \end{tikzpicture}

    Answer \theqcounter: What is encapsulation?%
  }

  {\footnotesize
  Encapsulation hides internal state by using access modifiers (\texttt{public}, \texttt{private}, \texttt{protected}, \texttt{internal}) and exposing behavior via methods/properties.
  }
\end{frame}

% 5 ── Loose coupling
\QuestionSlide[\CategoryBadge[OOPColor!20]{OOP}]{** What is loose coupling?}

\begin{frame}[fragile]
  \frametitle{%
    \begin{tikzpicture}[remember picture, overlay]
      \node[anchor=north east, xshift=-0.4cm, yshift=-0.4cm, text=black] at (current page.north east) {
        \CategoryBadge[OOPColor!20]{OOP}
      };
    \end{tikzpicture}

    Answer \theqcounter: What is loose coupling?%
  }

  {\footnotesize
  Loose coupling refers to a design principle where components or classes have little knowledge of the internal details of each other. This improves flexibility, makes code easier to maintain, test, and extend, and allows changes in one component without heavily impacting others.

  In .NET, loose coupling is often achieved using interfaces and dependency injection.
  }

    \begin{minted}[fontsize=\tiny]{csharp}
public interface ILogger
{
    void Log(string message);
}

public class ConsoleLogger : ILogger
{
    public void Log(string message) =>
        Console.WriteLine(message);
}

public class Processor
{
    private readonly ILogger _logger;

    public Processor(ILogger logger)
    {
        _logger = logger;
    }

    public void Run() => _logger.Log("Running...");
}
  \end{minted}
\end{frame}


\hypertarget{sec8}{}
\section{Design Patterns}
% 1 ── Dependency Injection in .NET
\QuestionSlide[\CategoryBadge[OOPColor!20]{Design Patterns}]{*** What is dependency injection in .NET?}

\begin{frame}[fragile]
  \frametitle{%
    \begin{tikzpicture}[remember picture, overlay]
      \node[anchor=north east, xshift=-0.4cm, yshift=-0.4cm, text=black] at (current page.north east) {
        \CategoryBadge[OOPColor!20]{Design Patterns}
      };
    \end{tikzpicture}

    Answer \theqcounter: What is dependency injection in .NET?%
  }

  {\footnotesize
  Dependency Injection (DI) is a design pattern where a class receives its dependencies from external sources rather than creating them internally. In .NET, this is commonly achieved via constructor injection and supported by the built-in DI container through \texttt{IServiceCollection}. DI promotes loose coupling, better testability, and easier maintenance.
  }

  \begin{minted}{csharp}
public interface IMessageService
{
    void Send(string message);
}

public class EmailService : IMessageService
{
    public void Send(string message)
    {
        Console.WriteLine($"Email: {message}");
    }
}

public class Notifier
{
    private readonly IMessageService _service;

    public Notifier(IMessageService service)
    {
        _service = service;
    }

    public void Notify(string msg) => _service.Send(msg);
}
  \end{minted}
\end{frame}

% 2 ── Exception handling design patterns
\QuestionSlide[\CategoryBadge[OOPColor!20]{Design Patterns}]{** What design pattern should be used for exception handling?}

\begin{frame}[fragile]
  \frametitle{%
    \begin{tikzpicture}[remember picture, overlay]
      \node[anchor=north east, xshift=-0.4cm, yshift=-0.4cm, text=black] at (current page.north east) {
        \CategoryBadge[OOPColor!20]{Design Patterns}
      };
    \end{tikzpicture}

    Answer \theqcounter: What design pattern should be used for exception handling?%
  }

  {\footnotesize
  While exception handling in C\# uses language constructs, certain design principles and patterns improve its structure and maintainability:
  \begin{itemize}
    \item \textbf{Command Pattern} – Encapsulate actions as objects to centralize error handling logic.
    \item \textbf{Strategy Pattern} – Provide interchangeable error-handling strategies based on context.
    \item \textbf{Separation of Concerns} – Keep core logic separate from error handling.
    \item \textbf{Fail-Fast Principle} – Validate early, throw early to avoid deep errors.
  \end{itemize}
  }
\end{frame}

% 3 ── Exception handling design pattern example
\begin{frame}[fragile]
  \frametitle{%
    \begin{tikzpicture}[remember picture, overlay]
      \node[anchor=north east, xshift=-0.4cm, yshift=-0.4cm, text=black] at (current page.north east) {
        \CategoryBadge[OOPColor!20]{Design Patterns}
      };
    \end{tikzpicture}

    Answer \theqcounter: Exception handling design pattern example%
  }

  \begin{minted}{csharp}
public interface IErrorHandler
{
    void Handle(Exception ex);
}

public class ConsoleErrorHandler : IErrorHandler
{
    public void Handle(Exception ex)
    {
        Console.WriteLine($"Error: {ex.Message}");
    }
}

try
{
    RunTask();
}
catch (Exception ex)
{
    IErrorHandler handler = new ConsoleErrorHandler();
    handler.Handle(ex);
}
  \end{minted}
\end{frame}

% 4 ── CQRS pattern
\QuestionSlide[\CategoryBadge[OOPColor!20]{Design Patterns}]{* What is the CQRS pattern?}

\begin{frame}[fragile]
  \frametitle{%
    \begin{tikzpicture}[remember picture, overlay]
      \node[anchor=north east, xshift=-0.4cm, yshift=-0.4cm, text=black] at (current page.north east) {
        \CategoryBadge[OOPColor!20]{Design Patterns}
      };
    \end{tikzpicture}

    Answer \theqcounter: What is CQRS?%
  }

  {\footnotesize
  Command Query Responsibility Segregation splits write (commands) and read (queries) models to optimise and scale each independently.
  }

  \begin{minted}{csharp}
// ICommand, IQuery handlers in separate namespaces
  \end{minted}
\end{frame}

% 5 ── Command–Query Separation (CQS)
\QuestionSlide[\CategoryBadge[OOPColor!20]{Design Patterns}]{* What is Command–Query Separation (CQS)?}

\begin{frame}[fragile]
  \frametitle{%
    \begin{tikzpicture}[remember picture, overlay]
      \node[anchor=north east, xshift=-0.4cm, yshift=-0.4cm, text=black] at (current page.north east) {
        \CategoryBadge[OOPColor!20]{Design Patterns}
      };
    \end{tikzpicture}

    Answer \theqcounter: What is CQS?%
  }

  {\footnotesize
  CQS states that a method should either perform an action (command) or return data (query) but never both.
  }

  \begin{minted}{csharp}
// Bad: SaveAndReturnStatus()
// Good: Save(); bool IsSaved();
  \end{minted}
\end{frame}

% 6 ── Value Object Pattern
\QuestionSlide[\CategoryBadge[OOPColor!20]{Design Patterns}]{*** What is the Value Object Pattern?}

\begin{frame}[fragile]
  \frametitle{%
    \begin{tikzpicture}[remember picture, overlay]
      \node[anchor=north east, xshift=-0.4cm, yshift=-0.4cm, text=black] at (current page.north east) {
        \CategoryBadge[OOPColor!20]{Design Patterns}
      };
    \end{tikzpicture}

    Answer \theqcounter: What is the Value Object Pattern?%
  }

  {\footnotesize
  The Value Object Pattern is a design pattern used to model objects that do not have a distinct identity. Instead, they are compared based on the values of their properties. Value Objects are typically immutable and used to represent concepts like money, dates, coordinates, or addresses.
  }

  \begin{minted}{csharp}
public class Address
{
    public string Street { get; }
    public string City { get; }

    public Address(string street, string city)
    {
        Street = street;
        City = city;
    }
}
  \end{minted}
\end{frame}

% 7 ── Why immutable Value Objects
\QuestionSlide[\CategoryBadge[OOPColor!20]{Design Patterns}]{*** Why should Value Objects be immutable?}

\begin{frame}[fragile]
  \frametitle{%
    \begin{tikzpicture}[remember picture, overlay]
      \node[anchor=north east, xshift=-0.4cm, yshift=-0.4cm, text=black] at (current page.north east) {
        \CategoryBadge[OOPColor!20]{Design Patterns}
      };
    \end{tikzpicture}

    Answer \theqcounter: Why should Value Objects be immutable?%
  }

  {\footnotesize
  Value Objects should be immutable because their identity is based on the combination of their property values. Changing any property would conceptually result in a different object. Immutability ensures consistency, thread-safety, and allows them to be safely reused and compared.
  }

  \begin{minted}{csharp}
var address1 = new Address("Main St", "Berlin");
// address1.City = "Paris"; // Not allowed - properties are read-only
  \end{minted}
\end{frame}

% 8 ── Comparing Value Objects
\QuestionSlide[\CategoryBadge[OOPColor!20]{Design Patterns}]{*** How do you compare Value Objects in C\#?}

\begin{frame}[fragile]
  \frametitle{%
    \begin{tikzpicture}[remember picture, overlay]
      \node[anchor=north east, xshift=-0.4cm, yshift=-0.4cm, text=black] at (current page.north east) {
        \CategoryBadge[OOPColor!20]{Design Patterns}
      };
    \end{tikzpicture}

    Answer \theqcounter: How do you compare Value Objects in C\#?%
  }

  {\footnotesize
  To compare Value Objects, override the \texttt{Equals} and \texttt{GetHashCode} methods. Also implement equality operators (\texttt{==} and \texttt{!=}) if needed. Comparisons should be based solely on property values.
  }

  \begin{minted}{csharp}
public override bool Equals(object obj)
{
    if (obj is not Address other) return false;
    return Street == other.Street && City == other.City;
}

public override int GetHashCode() =>
    HashCode.Combine(Street, City);
  \end{minted}
\end{frame}

% 9 ── When to use Value Object Pattern
\QuestionSlide[\CategoryBadge[OOPColor!20]{Design Patterns}]{*** When should you use the Value Object Pattern?}

\begin{frame}[fragile]
  \frametitle{%
    \begin{tikzpicture}[remember picture, overlay]
      \node[anchor=north east, xshift=-0.4cm, yshift=-0.4cm, text=black] at (current page.north east) {
        \CategoryBadge[OOPColor!20]{Design Patterns}
      };
    \end{tikzpicture}

    Answer \theqcounter: When should you use the Value Object Pattern?%
  }

  {\footnotesize
  Use the Value Object Pattern when:
  \begin{itemize}
    \item Objects are defined only by their values, not identity.
    \item You want safer equality comparisons.
    \item The data is conceptually atomic (e.g. money, coordinates).
    \item You need immutability and consistency across the domain.
  \end{itemize}
  Example: coordinates used as part of a game engine.
  }

  \begin{minted}{csharp}
public class Coordinates
{
    public int X { get; }
    public int Y { get; }

    public Coordinates(int x, int y)
    {
        X = x;
        Y = y;
    }
}
  \end{minted}
\end{frame}


\hypertarget{sec9}{}
\section{OAuth}
\QuestionSlide[\CategoryBadge[AuthColor!20]{Security}]{** What is OAuth 2.0 in the context of .NET/C\#?}

\begin{frame}[fragile]
  \frametitle{%
    \begin{tikzpicture}[remember picture, overlay]
      \node[anchor=north east, xshift=-0.4cm, yshift=-0.4cm, text=black] at (current page.north east) {
        \CategoryBadge[OOPColor!20]{C\#}
        \CategoryBadge[AuthColor!20]{Security}
      };
    \end{tikzpicture}
    Answer \theqcounter: What is OAuth 2.0 in the context of .NET/C\#?%
  }

  {\footnotesize
  OAuth 2.0 is a delegated authorization protocol: the client obtains a temporary access token (e.g., JWT) from the authorization server and uses it to call protected APIs, without ever handling the user’s password.
  }

  \begin{minted}[fontsize=\scriptsize]{csharp}
// Example JWT Bearer setup in .NET Core:
services.AddAuthentication("Bearer")
  .AddJwtBearer("Bearer", options => {
    options.Authority = "https://idp.example.com";
    options.Audience = "api1";
  });
  \end{minted}
\end{frame}

% Main roles in OAuth 2.0
\QuestionSlide[\CategoryBadge[AuthColor!20]{Security}]{**What are the main roles in OAuth 2.0?**}

\begin{frame}[fragile]
  \frametitle{%
    \begin{tikzpicture}[remember picture, overlay]
      \node[anchor=north east, xshift=-0.4cm, yshift=-0.4cm, text=black] at (current page.north east) {
        \CategoryBadge[OOPColor!20]{C\#}
        \CategoryBadge[AuthColor!20]{Security}
      };
    \end{tikzpicture}
    Answer \theqcounter: Main Roles in OAuth 2.0%
  }

  {\small
  \begin{itemize}
    \item \textbf{Resource Owner} – The \emph{user} (or system) who owns and grants access to their protected data.
    \item \textbf{Client} – The \emph{application} requesting access to the resource on behalf of the user.
    \item \textbf{Authorization Server} – Authenticates the resource owner, obtains their consent, and \textbf{issues access tokens}.
    \item \textbf{Resource Server} – Hosts the protected resources, \textbf{validates tokens}, and serves data to authorized clients.
  \end{itemize}

  \vspace{0.4cm}
  \begin{center}
  
\resizebox{0.92\linewidth}{!}{%
    \begin{tikzpicture}[node distance=2.2cm, >=latex, every node/.style={font=\scriptsize}]
      \node (owner) [draw, rounded corners, fill=AuthColor!15, minimum width=2.5cm, minimum height=0.8cm] {Resource Owner};
      \node (client) [draw, rounded corners, fill=blue!15, minimum width=2.5cm, minimum height=0.8cm, right=of owner] {Client};
      \node (auth)   [draw, rounded corners, fill=green!15, minimum width=3cm, minimum height=0.8cm, below=of owner, xshift=1.1cm] {Authorization Server};
      \node (rs)     [draw, rounded corners, fill=orange!15, minimum width=2.8cm, minimum height=0.8cm, right=of auth] {Resource Server};

      \draw[->] (owner) -- node[above]{Grants access} (client);
      \draw[->] (client) -- node[right]{Auth request} (auth);
      \draw[->] (auth) -- node[left]{Token} (client);
      \draw[->] (client) -- node[below]{API request w/ token} (rs);
      \draw[->] (rs) -- node[below]{Protected data} (client);
    \end{tikzpicture}
}
\end{center}
  }
\end{frame}


% PKCE
\QuestionSlide[\CategoryBadge[AuthColor!20]{Security}]{** What is PKCE and why is it used?}

\begin{frame}[fragile]
  \frametitle{%
    \begin{tikzpicture}[remember picture, overlay]
      \node[anchor=north east, xshift=-0.4cm, yshift=-0.4cm, text=black] at (current page.north east) {
        \CategoryBadge[OOPColor!20]{C\#}
        \CategoryBadge[AuthColor!20]{Security}
      };
    \end{tikzpicture}
    Answer \theqcounter: What is PKCE and why is it used?%
  }

  {\footnotesize
  PKCE (Proof Key for Code Exchange) secures the Authorization Code Flow for public clients (e.g., SPAs, mobile apps) by having the client send a `code verifier` and `code challenge`, preventing code interception attacks.
  }
\end{frame}

% Authorization Code Flow
\QuestionSlide[\CategoryBadge[AuthColor!20]{Security}]{What is the Authorization Code Flow?}

\begin{frame}[fragile]
  \frametitle{%
    \begin{tikzpicture}[remember picture, overlay]
      \node[anchor=north east, xshift=-0.4cm, yshift=-0.4cm, text=black] at (current page.north east) {
        \CategoryBadge[OOPColor!20]{C\#}
        \CategoryBadge[AuthColor!20]{Security}
      };
    \end{tikzpicture}
    Answer \theqcounter: What is the Authorization Code Flow?%
  }

  {\footnotesize
  A four-step redirect-based protocol where the client trades an authorization code for tokens without ever seeing the user's password.
  }
\end{frame}

\QuestionSlide[\CategoryBadge[AuthColor!20]{Security}]{Authorization Code Flow — Step 1}

\begin{frame}[fragile]
  \frametitle{%
    \begin{tikzpicture}[remember picture, overlay]
      \node[anchor=north east, xshift=-0.4cm, yshift=-0.4cm, text=black] at (current page.north east) {
        \CategoryBadge[OOPColor!20]{C\#}
        \CategoryBadge[AuthColor!20]{Security}
      };
    \end{tikzpicture}
    Answer \theqcounter: Step 1%
  }

  {\footnotesize
  The client redirects the user's browser to the IdP's \texttt{/authorize} endpoint with \texttt{client\_id}, \texttt{redirect\_uri}, and a PKCE \texttt{code\_challenge}.
  }
\end{frame}

\QuestionSlide[\CategoryBadge[AuthColor!20]{Security}]{Authorization Code Flow — Step 2}

\begin{frame}[fragile]
  \frametitle{%
    \begin{tikzpicture}[remember picture, overlay]
      \node[anchor=north east, xshift=-0.4cm, yshift=-0.4cm, text=black] at (current page.north east) {
        \CategoryBadge[OOPColor!20]{C\#}
        \CategoryBadge[AuthColor!20]{Security}
      };
    \end{tikzpicture}
    Answer \theqcounter: Step 2%
  }

  {\footnotesize
  The Identity Provider authenticates the user, obtains consent, and redirects back to the client with an authorization \texttt{code}.
  }
\end{frame}

\QuestionSlide[\CategoryBadge[AuthColor!20]{Security}]{Authorization Code Flow — Step 3}

\begin{frame}[fragile]
  \frametitle{%
    \begin{tikzpicture}[remember picture, overlay]
      \node[anchor=north east, xshift=-0.4cm, yshift=-0.4cm, text=black] at (current page.north east) {
        \CategoryBadge[OOPColor!20]{C\#}
        \CategoryBadge[AuthColor!20]{Security}
      };
    \end{tikzpicture}
    Answer \theqcounter: Step 3%
  }

  {\footnotesize
  The client posts the \texttt{code} and PKCE \texttt{code\_verifier} to the \texttt{/token} endpoint to obtain an access token (and optionally refresh or ID tokens).
  }
\end{frame}

\QuestionSlide[\CategoryBadge[AuthColor!20]{Security}]{Authorization Code Flow — Step 4}

\begin{frame}[fragile]
  \frametitle{%
    \begin{tikzpicture}[remember picture, overlay]
      \node[anchor=north east, xshift=-0.4cm, yshift=-0.4cm, text=black] at (current page.north east) {
        \CategoryBadge[OOPColor!20]{C\#}
        \CategoryBadge[AuthColor!20]{Security}
      };
    \end{tikzpicture}
    Answer \theqcounter: Step 4%
  }

  {\footnotesize
  The client calls the API using \texttt{Authorization: Bearer <access\_token>} and may use refresh tokens when the access token expires.
  }
\end{frame}

\QuestionSlide[\CategoryBadge[AuthColor!20]{Security}]{Authorization Code Flow Diagram}

\begin{frame}
  \frametitle{Authorization Code Flow Diagram}

  \vspace{0.2cm}

  \begin{center}
  \resizebox{0.95\linewidth}{!}{%
  \begin{tikzpicture}[>=Latex, font=\scriptsize]
    % Lifeline heads
    \node (user)   at (0,0)   {User Browser};
    \node (client) at (4,0)   {Client App};
    \node (idp)    at (8,0)   {Identity Provider};
    \node (api)    at (12,0)  {API};

    % Lifelines
    \draw[dashed] (user) -- ++(0,-6.2);
    \draw[dashed] (client) -- ++(0,-6.2);
    \draw[dashed] (idp) -- ++(0,-6.2);
    \draw[dashed] (api) -- ++(0,-6.2);

    % y-levels for steps
    \coordinate (y1) at (0,-0.9);
    \coordinate (y2) at (0,-1.8);
    \coordinate (y3) at (0,-2.7);
    \coordinate (y4) at (0,-3.6);
    \coordinate (y5) at (0,-4.5);
    \coordinate (y6) at (0,-5.4);

    % 1) Client -> Browser: redirect to /authorize
    \draw[->] (client |- y1) -- (user |- y1)
      node[midway, above]{302 redirect to \texttt{/authorize}};

    % 2) Browser -> IdP: /authorize (with PKCE challenge/state)
    \draw[->] (user |- y2) -- (idp |- y2)
      node[midway, above]{GET \texttt{/authorize} (+ code\_challenge, state)};

    % 3) IdP -> Browser: login/consent UI
    \draw[->] (idp |- y3) -- (user |- y3)
      node[midway, above]{Login \& Consent};

    % 4) Browser -> Client: redirect back with code
    \draw[->] (user |- y4) -- (client |- y4)
      node[midway, above]{302 to \texttt{/callback?code=\dots}};

    % 5) Client -> IdP: exchange code for tokens (PKCE)
    \draw[->] (client |- y5) -- (idp |- y5)
      node[midway, above]{POST \texttt{/token} (code\_verifier)};

    % 6) IdP -> Client: access token
    \draw[->] (idp |- y6) -- (client |- y6)
      node[midway, above]{Access token};

      % 7) Client -> API
    \coordinate (y7) at (0,-6.0);
    \coordinate (y8) at (0,-6.9);

    \draw[->] (client |- y7) -- (api |- y7)
      node[midway, above]{\texttt{Authorization: Bearer <token>}};

    % 8) API -> Client
    \draw[->] (api |- y8) -- (client |- y8)
      node[midway, above]{API response};

  \end{tikzpicture}%
  }
  \end{center}
\end{frame}

% Client Credentials Flow
\QuestionSlide[\CategoryBadge[AuthColor!20]{Security}]{** What is the Client Credentials Flow?}

\begin{frame}[fragile]
  \frametitle{%
    \begin{tikzpicture}[remember picture, overlay]
      \node[anchor=north east, xshift=-0.4cm, yshift=-0.4cm, text=black] at (current page.north east) {
        \CategoryBadge[OOPColor!20]{C\#}
        \CategoryBadge[AuthColor!20]{Security}
      };
    \end{tikzpicture}
    Answer \theqcounter: What is the Client Credentials Flow?%
  }

  {\footnotesize
  A machine-to-machine flow where the client uses its own Client ID and Secret to obtain an access token, without user involvement. Ideal for server-to-server integrations.
  }
\end{frame}

% Scopes
\QuestionSlide[\CategoryBadge[AuthColor!20]{Security}]{** What are scopes in OAuth 2.0?}

\begin{frame}[fragile]
  \frametitle{%
    \begin{tikzpicture}[remember picture, overlay]
      \node[anchor=north east, xshift=-0.4cm, yshift=-0.4cm, text=black] at (current page.north east) {
        \CategoryBadge[OOPColor!20]{C\#}
        \CategoryBadge[AuthColor!20]{Security}
      };
    \end{tikzpicture}
    Answer \theqcounter: What are scopes in OAuth 2.0?%
  }

  {\footnotesize
  Scopes are named permissions (e.g., `read`, `write`) that the client requests. The issued token contains approved scopes, which the Resource Server enforces.
  }
\end{frame}

% JWT Bearer configuration in ASP.NET Core
\QuestionSlide[\CategoryBadge[AuthColor!20]{Security}]{** How do you configure JWT Bearer Authentication in ASP\.NET Core?}

\begin{frame}[fragile]
  \frametitle{%
    \begin{tikzpicture}[remember picture, overlay]
      \node[anchor=north east, xshift=-0.4cm, yshift=-0.4cm, text=black] at (current page.north east) {
        \CategoryBadge[OOPColor!20]{C\#}
        \CategoryBadge[AuthColor!20]{Security}
      };
    \end{tikzpicture}
    Answer \theqcounter: How do you configure JWT Bearer Authentication in ASP\.NET Core?%
  }

  {\footnotesize
  Use the `JwtBearer` middleware in `Startup`/`Program.cs`, specifying `Authority` and `Audience`:
  }

  \begin{minted}[fontsize=\scriptsize]{csharp}
services.AddAuthentication(JwtBearerDefaults.AuthenticationScheme)
  .AddJwtBearer(options => {
    options.Authority = "https://idp.example.com";
    options.Audience = "api1";
    options.RequireHttpsMetadata = true;
  });
  \end{minted}
\end{frame}

% OpenID Connect – what is it?
\QuestionSlide[\CategoryBadge[AuthColor!20]{Security}]{** What is OpenID Connect?}

\begin{frame}[fragile]
  \frametitle{%
    \begin{tikzpicture}[remember picture, overlay]
      \node[anchor=north east, xshift=-0.4cm, yshift=-0.4cm, text=black] at (current page.north east) {
        \CategoryBadge[OOPColor!20]{C\#}
        \CategoryBadge[AuthColor!20]{Security}
      };
    \end{tikzpicture}
    Answer \theqcounter: What is OpenID Connect?%
  }

  {\footnotesize
  OpenID Connect is an authentication layer built on OAuth 2.0. It adds an ID Token (JWT) containing user identity claims, plus discovery and UserInfo endpoints.
  }
\end{frame}

% ID Token
\QuestionSlide[\CategoryBadge[AuthColor!20]{Security}]{** What is an ID Token?}

\begin{frame}[fragile]
  \frametitle{%
    \begin{tikzpicture}[remember picture, overlay]
      \node[anchor=north east, xshift=-0.4cm, yshift=-0.4cm, text=black] at (current page.north east) {
        \CategoryBadge[OOPColor!20]{C\#}
        \CategoryBadge[AuthColor!20]{Security}
      };
    \end{tikzpicture}
    Answer \theqcounter: What is an ID Token?%
  }

  {\footnotesize
  An ID Token is a JWT issued by the OpenID Provider during the Authorization Code Flow, containing user identity claims such as `sub`, `email`, and `name`.
  }
\end{frame}

% OIDC configuration in ASP.NET Core
\QuestionSlide[\CategoryBadge[AuthColor!20]{Security}]{** How do you implement OpenID Connect in ASP\.NET Core?}

\begin{frame}[fragile]
  \frametitle{%
    \begin{tikzpicture}[remember picture, overlay]
      \node[anchor=north east, xshift=-0.4cm, yshift=-0.4cm, text=black] at (current page.north east) {
        \CategoryBadge[OOPColor!20]{C\#}
        \CategoryBadge[AuthColor!20]{Security}
      };
    \end{tikzpicture}
    Answer \theqcounter: How do you implement OpenID Connect in ASP\.NET Core?%
  }

  {\footnotesize
  Configure the OIDC scheme in `Program.cs`:
  }

  \begin{minted}[fontsize=\scriptsize]{csharp}
services.AddAuthentication(options => {
  options.DefaultScheme = "Cookies";
  options.DefaultChallengeScheme = "oidc";
})
.AddCookie("Cookies")
.AddOpenIdConnect("oidc", options => {
  options.Authority = "https://idp.example.com";
  options.ClientId = "mvc_client";
  options.ClientSecret = "secret";
  options.ResponseType = "code";
  options.Scope.Add("openid");
  options.Scope.Add("profile");
  options.SaveTokens = true;
});
  \end{minted}
\end{frame}

% OAuth 2.0 vs OpenID Connect
\QuestionSlide[\CategoryBadge[AuthColor!20]{Security}]{** What is the difference between OAuth 2.0 and OpenID Connect?}

\begin{frame}[fragile]
  \frametitle{%
    \begin{tikzpicture}[remember picture, overlay]
      \node[anchor=north east, xshift=-0.4cm, yshift=-0.4cm, text=black] at (current page.north east) {
        \CategoryBadge[OOPColor!20]{C\#}
        \CategoryBadge[AuthColor!20]{Security}
      };
    \end{tikzpicture}
    Answer \theqcounter: What is the difference between OAuth 2.0 and OpenID Connect?%
  }

  {\footnotesize
  OAuth 2.0 is a protocol for delegating access (authorization).\\
  OpenID Connect extends OAuth 2.0 by adding authentication, issuing an ID Token that verifies the user’s identity.
  }
\end{frame}

% 1 ── ASP.NET Core Identity
\QuestionSlide[\CategoryBadge[AuthColor!20]{Security}]{** What is ASP\.NET Core Identity?}

\begin{frame}[fragile]
  \frametitle{%
    \begin{tikzpicture}[remember picture, overlay]
      \node[anchor=north east, xshift=-0.4cm, yshift=-0.4cm, text=black] at (current page.north east) {
        \CategoryBadge[OOPColor!20]{C\#}
        \CategoryBadge[AuthColor!20]{Security}
      };
    \end{tikzpicture}
    Answer \theqcounter: What is ASP\.NET Core Identity?%
  }

  {\footnotesize
  ASP\.NET Core Identity is a membership system that adds login, logout, user management, roles, claims, and storing user data (via EF Core by default) to your application.
  }

  \begin{minted}{csharp}
// In Program.cs
services.AddDefaultIdentity<IdentityUser>()
    .AddEntityFrameworkStores<ApplicationDbContext>();
  \end{minted}
\end{frame}

% 2 ── Cookie Authentication
\QuestionSlide[\CategoryBadge[AuthColor!20]{Security}]{** What is Cookie Authentication in ASP\.NET Core?}

\begin{frame}[fragile]
  \frametitle{%
    \begin{tikzpicture}[remember picture, overlay]
      \node[anchor=north east, xshift=-0.4cm, yshift=-0.4cm, text=black] at (current page.north east) {
        \CategoryBadge[OOPColor!20]{C\#}
        \CategoryBadge[AuthColor!20]{Security}
      };
    \end{tikzpicture}
    Answer \theqcounter: What is Cookie Authentication in ASP\.NET Core?%
  }

  {\footnotesize
  Cookie Authentication issues an encrypted cookie after login, which is sent on each request to identify the user session. It’s suitable for web apps that maintain user sessions.
  }

  \begin{minted}{csharp}
services.AddAuthentication(CookieAuthenticationDefaults.AuthenticationScheme)
    .AddCookie(options => {
      options.LoginPath = "/Account/Login";
      options.ExpireTimeSpan = TimeSpan.FromHours(1);
    });
  \end{minted}
\end{frame}

% 3 ── Claims-based Authorization
\QuestionSlide[\CategoryBadge[AuthColor!20]{Security}]{** What is Claims-based Authorization?}

\begin{frame}[fragile]
  \frametitle{%
    \begin{tikzpicture}[remember picture, overlay]
      \node[anchor=north east, xshift=-0.4cm, yshift=-0.4cm, text=black] at (current page.north east) {
        \CategoryBadge[OOPColor!20]{C\#}
        \CategoryBadge[AuthColor!20]{Security}
      };
    \end{tikzpicture}
    Answer \theqcounter: What is Claims-based Authorization?%
  }

  {\footnotesize
  Claims-based Authorization grants or denies access based on claims in the user’s identity (e.g., roles, permissions). Policies evaluate claims via requirements and handlers.
  }

  \begin{minted}{csharp}
// Define policy
services.AddAuthorization(options => {
  options.AddPolicy("CanEdit", policy =>
    policy.RequireClaim("Permission", "Edit"));
});

// Enforce in controller
[Authorize(Policy = "CanEdit")]
public IActionResult Edit() => View();
  \end{minted}
\end{frame}

% 4 ── Roles vs Policies
\QuestionSlide[\CategoryBadge[AuthColor!20]{Security}]{** What is the difference between Roles and Policies?}

\begin{frame}[fragile]
  \frametitle{%
    \begin{tikzpicture}[remember picture, overlay]
      \node[anchor=north east, xshift=-0.4cm, yshift=-0.4cm, text=black] at (current page.north east) {
        \CategoryBadge[OOPColor!20]{C\#}
        \CategoryBadge[AuthColor!20]{Security}
      };
    \end{tikzpicture}
    Answer \theqcounter: What is the difference between Roles and Policies?%
  }

  {\footnotesize
  Roles assign users to named groups (e.g., “Admin”). Policies are more flexible rules combining multiple claims, roles, or custom requirements.
  }

  \begin{minted}{csharp}
// Role-based
[Authorize(Roles = "Admin")]

// Policy-based
[Authorize(Policy = "CanEdit")]
  \end{minted}
\end{frame}

% 5 ── Refresh Tokens in .NET
\QuestionSlide[\CategoryBadge[AuthColor!20]{Security}]{** What is a Refresh Token and how is it used?}

\begin{frame}[fragile]
  \frametitle{%
    \begin{tikzpicture}[remember picture, overlay]
      \node[anchor=north east, xshift=-0.4cm, yshift=-0.4cm, text=black] at (current page.north east) {
        \CategoryBadge[OOPColor!20]{C\#}
        \CategoryBadge[AuthColor!20]{Security}
      };
    \end{tikzpicture}
    Answer \theqcounter: What is a Refresh Token and how is it used?%
  }

  {\footnotesize
  A Refresh Token is a long-lived token used to obtain new access tokens when they expire. It’s stored securely and sent to the token endpoint in exchange for fresh tokens.
  }

  \begin{minted}{csharp}
// Exchange refresh token
var tokenResponse = await httpClient.RequestRefreshTokenAsync(new RefreshTokenRequest {
  Address = "https://idp.example.com/connect/token",
  ClientId = "client",
  ClientSecret = "secret",
  RefreshToken = storedRefreshToken
});
  \end{minted}
\end{frame}


\hypertarget{sec10}{}
\section{Microservices}
% Microservices section

% 1 – Bulkhead Pattern
\QuestionSlide[\CategoryBadge[MicroColor!20]{Microservices}]{** What is the Bulkhead Pattern?}

\begin{frame}[fragile]
  \frametitle{%
    \begin{tikzpicture}[remember picture, overlay]
      \node[anchor=north east, xshift=-0.4cm, yshift=-0.4cm, text=black] at (current page.north east) {
        \CategoryBadge[MicroColor!20]{Microservices}
      };
    \end{tikzpicture}
    Answer \theqcounter: What is the Bulkhead Pattern?%
  }

  {\footnotesize
  The Bulkhead Pattern isolates critical resources (threads, connections) into independent pools so that a failure in one area does not bring down the entire system.
  }

  \begin{minted}[fontsize=\scriptsize]{csharp}
var bulkhead = 
Policy.BulkheadAsync<HttpResponseMessage>(
    maxParallelization: 50,
    maxQueuingActions: 100);
await bulkhead.ExecuteAsync(() => httpClient.GetAsync(url));
  \end{minted}
\end{frame}

% 2 – Sidecar Pattern
\QuestionSlide[\CategoryBadge[MicroColor!20]{Microservices}]{* What is the Sidecar Pattern?}

\begin{frame}[fragile]
  \frametitle{%
    \begin{tikzpicture}[remember picture, overlay]
      \node[anchor=north east, xshift=-0.4cm, yshift=-0.4cm, text=black] at (current page.north east) {
        \CategoryBadge[MicroColor!20]{Microservices}
      };
    \end{tikzpicture}
    Answer \theqcounter: What is the Sidecar Pattern?%
  }

  {\footnotesize
  The Sidecar Pattern deploys a helper service alongside the main service within the same host or pod. The sidecar handles cross-cutting concerns like logging or proxying without modifying the primary service.
  }

  \begin{minted}[fontsize=\scriptsize]{yaml}
apiVersion: v1
kind: Pod
metadata:
  name: app
spec:
  containers:
  - name: web
    image: myapi:1.0
  - name: logger
    image: fluentd:latest
  \end{minted}
\end{frame}

% 3 – Service Mesh
\QuestionSlide[\CategoryBadge[MicroColor!20]{Microservices}]{** What is a Service Mesh?}

\begin{frame}[fragile]
  \frametitle{%
    \begin{tikzpicture}[remember picture, overlay]
      \node[anchor=north east, xshift=-0.4cm, yshift=-0.4cm, text=black] at (current page.north east) {
        \CategoryBadge[MicroColor!20]{Microservices}
      };
    \end{tikzpicture}
    Answer \theqcounter: What is a Service Mesh?%
  }

  {\footnotesize
  A Service Mesh is an infrastructure layer that manages service-to-service communication via sidecar proxies, providing features like traffic routing, retries, and observability without changing application code. Popular meshes include Istio and Linkerd.
  }

  \begin{minted}[fontsize=\scriptsize]{bash}
# Enable Istio sidecar injection
kubectl label namespace default istio-injection=enabled
  \end{minted}
\end{frame}

% 4 – Strangler Fig Pattern
\QuestionSlide[\CategoryBadge[MicroColor!20]{Microservices}]{* What is the Strangler Fig Pattern?}

\begin{frame}[fragile]
  \frametitle{%
    \begin{tikzpicture}[remember picture, overlay]
      \node[anchor=north east, xshift=-0.4cm, yshift=-0.4cm, text=black] at (current page.north east) {
        \CategoryBadge[MicroColor!20]{Microservices}
      };
    \end{tikzpicture}
    Answer \theqcounter: What is the Strangler Fig Pattern?%
  }

  {\footnotesize
  The Strangler Fig Pattern incrementally replaces pieces of a monolith with microservices. A facade routes requests between legacy and new components until the monolith can be retired.
  }

  \begin{minted}[fontsize=\scriptsize]{csharp}
// ASP.NET Core facade routing
app.MapWhen(ctx => ctx.Request.Path.StartsWithSegments("/legacy"),
    legacy => legacy.RunProxy("http://monolith"));
app.Map("/orders", m => m.RunProxy("http://orders-service"));
  \end{minted}
\end{frame}

% 5 – Event Sourcing
\QuestionSlide[\CategoryBadge[MicroColor!20]{Microservices}]{** What is Event Sourcing?}

\begin{frame}[fragile]
  \frametitle{%
    \begin{tikzpicture}[remember picture, overlay]
      \node[anchor=north east, xshift=-0.4cm, yshift=-0.4cm, text=black] at (current page.north east) {
        \CategoryBadge[MicroColor!20]{Microservices}
      };
    \end{tikzpicture}
    Answer \theqcounter: What is Event Sourcing?%
  }

  {\footnotesize
  Event Sourcing persists state as a sequence of immutable events. The current state is rebuilt by replaying events, enabling full audit history and temporal queries.
  }

  \begin{minted}[fontsize=\scriptsize]{csharp}
public record FundsDeposited(Guid AccountId, decimal Amount);

public class Account {
    private decimal _balance;
    public void Apply(FundsDeposited e) => _balance += e.Amount;
}
  \end{minted}
\end{frame}


\begin{frame}
  \centering
  {\Huge Thank you!}\\[0.5em]
  Happy learning and coding.
\end{frame}

\end{document}

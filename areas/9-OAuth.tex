\QuestionSlide[\CategoryBadge[AuthColor!20]{Security}]{** What is OAuth 2.0 in the context of .NET/C\#?}

\begin{frame}[fragile]
  \frametitle{%
    \begin{tikzpicture}[remember picture, overlay]
      \node[anchor=north east, xshift=-0.4cm, yshift=-0.4cm, text=black] at (current page.north east) {
        \CategoryBadge[OOPColor!20]{C\#}
        \CategoryBadge[AuthColor!20]{Security}
      };
    \end{tikzpicture}
    Answer \theqcounter: What is OAuth 2.0 in the context of .NET/C\#?%
  }

  {\footnotesize
  OAuth 2.0 is a delegated authorization protocol: the client obtains a temporary access token (e.g., JWT) from the authorization server and uses it to call protected APIs, without ever handling the user’s password.
  }

  \begin{minted}[fontsize=\scriptsize]{csharp}
// Example JWT Bearer setup in .NET Core:
services.AddAuthentication("Bearer")
  .AddJwtBearer("Bearer", options => {
    options.Authority = "https://idp.example.com";
    options.Audience = "api1";
  });
  \end{minted}
\end{frame}

% Main roles in OAuth 2.0
\QuestionSlide[\CategoryBadge[AuthColor!20]{Security}]{**What are the main roles in OAuth 2.0?**}

\begin{frame}[fragile]
  \frametitle{%
    \begin{tikzpicture}[remember picture, overlay]
      \node[anchor=north east, xshift=-0.4cm, yshift=-0.4cm, text=black] at (current page.north east) {
        \CategoryBadge[OOPColor!20]{C\#}
        \CategoryBadge[AuthColor!20]{Security}
      };
    \end{tikzpicture}
    Answer \theqcounter: Main Roles in OAuth 2.0%
  }

  {\small
  \begin{itemize}
    \item \textbf{Resource Owner} – The \emph{user} (or system) who owns and grants access to their protected data.
    \item \textbf{Client} – The \emph{application} requesting access to the resource on behalf of the user.
    \item \textbf{Authorization Server} – Authenticates the resource owner, obtains their consent, and \textbf{issues access tokens}.
    \item \textbf{Resource Server} – Hosts the protected resources, \textbf{validates tokens}, and serves data to authorized clients.
  \end{itemize}

  \vspace{0.4cm}
  \begin{center}
  
\resizebox{0.92\linewidth}{!}{%
    \begin{tikzpicture}[node distance=2.2cm, >=latex, every node/.style={font=\scriptsize}]
      \node (owner) [draw, rounded corners, fill=AuthColor!15, minimum width=2.5cm, minimum height=0.8cm] {Resource Owner};
      \node (client) [draw, rounded corners, fill=blue!15, minimum width=2.5cm, minimum height=0.8cm, right=of owner] {Client};
      \node (auth)   [draw, rounded corners, fill=green!15, minimum width=3cm, minimum height=0.8cm, below=of owner, xshift=1.1cm] {Authorization Server};
      \node (rs)     [draw, rounded corners, fill=orange!15, minimum width=2.8cm, minimum height=0.8cm, right=of auth] {Resource Server};

      \draw[->] (owner) -- node[above]{Grants access} (client);
      \draw[->] (client) -- node[right]{Auth request} (auth);
      \draw[->] (auth) -- node[left]{Token} (client);
      \draw[->] (client) -- node[below]{API request w/ token} (rs);
      \draw[->] (rs) -- node[below]{Protected data} (client);
    \end{tikzpicture}
}
\end{center}
  }
\end{frame}


% PKCE
\QuestionSlide[\CategoryBadge[AuthColor!20]{Security}]{** What is PKCE and why is it used?}

\begin{frame}[fragile]
  \frametitle{%
    \begin{tikzpicture}[remember picture, overlay]
      \node[anchor=north east, xshift=-0.4cm, yshift=-0.4cm, text=black] at (current page.north east) {
        \CategoryBadge[OOPColor!20]{C\#}
        \CategoryBadge[AuthColor!20]{Security}
      };
    \end{tikzpicture}
    Answer \theqcounter: What is PKCE and why is it used?%
  }

  {\footnotesize
  PKCE (Proof Key for Code Exchange) secures the Authorization Code Flow for public clients (e.g., SPAs, mobile apps) by having the client send a `code verifier` and `code challenge`, preventing code interception attacks.
  }
\end{frame}

% Authorization Code Flow
\QuestionSlide[\CategoryBadge[AuthColor!20]{Security}]{What is the Authorization Code Flow?}

\begin{frame}[fragile]
  \frametitle{%
    \begin{tikzpicture}[remember picture, overlay]
      \node[anchor=north east, xshift=-0.4cm, yshift=-0.4cm, text=black] at (current page.north east) {
        \CategoryBadge[OOPColor!20]{C\#}
        \CategoryBadge[AuthColor!20]{Security}
      };
    \end{tikzpicture}
    Answer \theqcounter: What is the Authorization Code Flow?%
  }

  {\footnotesize
  A four-step redirect-based protocol where the client trades an authorization code for tokens without ever seeing the user's password.
  }
\end{frame}

\QuestionSlide[\CategoryBadge[AuthColor!20]{Security}]{Authorization Code Flow — Step 1}

\begin{frame}[fragile]
  \frametitle{%
    \begin{tikzpicture}[remember picture, overlay]
      \node[anchor=north east, xshift=-0.4cm, yshift=-0.4cm, text=black] at (current page.north east) {
        \CategoryBadge[OOPColor!20]{C\#}
        \CategoryBadge[AuthColor!20]{Security}
      };
    \end{tikzpicture}
    Answer \theqcounter: Step 1%
  }

  {\footnotesize
  The client redirects the user's browser to the IdP's \texttt{/authorize} endpoint with \texttt{client\_id}, \texttt{redirect\_uri}, and a PKCE \texttt{code\_challenge}.
  }
\end{frame}

\QuestionSlide[\CategoryBadge[AuthColor!20]{Security}]{Authorization Code Flow — Step 2}

\begin{frame}[fragile]
  \frametitle{%
    \begin{tikzpicture}[remember picture, overlay]
      \node[anchor=north east, xshift=-0.4cm, yshift=-0.4cm, text=black] at (current page.north east) {
        \CategoryBadge[OOPColor!20]{C\#}
        \CategoryBadge[AuthColor!20]{Security}
      };
    \end{tikzpicture}
    Answer \theqcounter: Step 2%
  }

  {\footnotesize
  The Identity Provider authenticates the user, obtains consent, and redirects back to the client with an authorization \texttt{code}.
  }
\end{frame}

\QuestionSlide[\CategoryBadge[AuthColor!20]{Security}]{Authorization Code Flow — Step 3}

\begin{frame}[fragile]
  \frametitle{%
    \begin{tikzpicture}[remember picture, overlay]
      \node[anchor=north east, xshift=-0.4cm, yshift=-0.4cm, text=black] at (current page.north east) {
        \CategoryBadge[OOPColor!20]{C\#}
        \CategoryBadge[AuthColor!20]{Security}
      };
    \end{tikzpicture}
    Answer \theqcounter: Step 3%
  }

  {\footnotesize
  The client posts the \texttt{code} and PKCE \texttt{code\_verifier} to the \texttt{/token} endpoint to obtain an access token (and optionally refresh or ID tokens).
  }
\end{frame}

\QuestionSlide[\CategoryBadge[AuthColor!20]{Security}]{Authorization Code Flow — Step 4}

\begin{frame}[fragile]
  \frametitle{%
    \begin{tikzpicture}[remember picture, overlay]
      \node[anchor=north east, xshift=-0.4cm, yshift=-0.4cm, text=black] at (current page.north east) {
        \CategoryBadge[OOPColor!20]{C\#}
        \CategoryBadge[AuthColor!20]{Security}
      };
    \end{tikzpicture}
    Answer \theqcounter: Step 4%
  }

  {\footnotesize
  The client calls the API using \texttt{Authorization: Bearer <access\_token>} and may use refresh tokens when the access token expires.
  }
\end{frame}

\QuestionSlide[\CategoryBadge[AuthColor!20]{Security}]{Authorization Code Flow Diagram}

\begin{frame}
  \frametitle{Authorization Code Flow Diagram}

  \vspace{0.2cm}

  \begin{center}
  \resizebox{0.95\linewidth}{!}{%
  \begin{tikzpicture}[>=Latex, font=\scriptsize]
    % Lifeline heads
    \node (user)   at (0,0)   {User Browser};
    \node (client) at (4,0)   {Client App};
    \node (idp)    at (8,0)   {Identity Provider};
    \node (api)    at (12,0)  {API};

    % Lifelines
    \draw[dashed] (user) -- ++(0,-6.2);
    \draw[dashed] (client) -- ++(0,-6.2);
    \draw[dashed] (idp) -- ++(0,-6.2);
    \draw[dashed] (api) -- ++(0,-6.2);

    % y-levels for steps
    \coordinate (y1) at (0,-0.9);
    \coordinate (y2) at (0,-1.8);
    \coordinate (y3) at (0,-2.7);
    \coordinate (y4) at (0,-3.6);
    \coordinate (y5) at (0,-4.5);
    \coordinate (y6) at (0,-5.4);

    % 1) Client -> Browser: redirect to /authorize
    \draw[->] (client |- y1) -- (user |- y1)
      node[midway, above]{302 redirect to \texttt{/authorize}};

    % 2) Browser -> IdP: /authorize (with PKCE challenge/state)
    \draw[->] (user |- y2) -- (idp |- y2)
      node[midway, above]{GET \texttt{/authorize} (+ code\_challenge, state)};

    % 3) IdP -> Browser: login/consent UI
    \draw[->] (idp |- y3) -- (user |- y3)
      node[midway, above]{Login \& Consent};

    % 4) Browser -> Client: redirect back with code
    \draw[->] (user |- y4) -- (client |- y4)
      node[midway, above]{302 to \texttt{/callback?code=\dots}};

    % 5) Client -> IdP: exchange code for tokens (PKCE)
    \draw[->] (client |- y5) -- (idp |- y5)
      node[midway, above]{POST \texttt{/token} (code\_verifier)};

    % 6) IdP -> Client: access token
    \draw[->] (idp |- y6) -- (client |- y6)
      node[midway, above]{Access token};

      % 7) Client -> API
    \coordinate (y7) at (0,-6.0);
    \coordinate (y8) at (0,-6.9);

    \draw[->] (client |- y7) -- (api |- y7)
      node[midway, above]{\texttt{Authorization: Bearer <token>}};

    % 8) API -> Client
    \draw[->] (api |- y8) -- (client |- y8)
      node[midway, above]{API response};

  \end{tikzpicture}%
  }
  \end{center}
\end{frame}

% Client Credentials Flow
\QuestionSlide[\CategoryBadge[AuthColor!20]{Security}]{** What is the Client Credentials Flow?}

\begin{frame}[fragile]
  \frametitle{%
    \begin{tikzpicture}[remember picture, overlay]
      \node[anchor=north east, xshift=-0.4cm, yshift=-0.4cm, text=black] at (current page.north east) {
        \CategoryBadge[OOPColor!20]{C\#}
        \CategoryBadge[AuthColor!20]{Security}
      };
    \end{tikzpicture}
    Answer \theqcounter: What is the Client Credentials Flow?%
  }

  {\footnotesize
  A machine-to-machine flow where the client uses its own Client ID and Secret to obtain an access token, without user involvement. Ideal for server-to-server integrations.
  }
\end{frame}

% Scopes
\QuestionSlide[\CategoryBadge[AuthColor!20]{Security}]{** What are scopes in OAuth 2.0?}

\begin{frame}[fragile]
  \frametitle{%
    \begin{tikzpicture}[remember picture, overlay]
      \node[anchor=north east, xshift=-0.4cm, yshift=-0.4cm, text=black] at (current page.north east) {
        \CategoryBadge[OOPColor!20]{C\#}
        \CategoryBadge[AuthColor!20]{Security}
      };
    \end{tikzpicture}
    Answer \theqcounter: What are scopes in OAuth 2.0?%
  }

  {\footnotesize
  Scopes are named permissions (e.g., `read`, `write`) that the client requests. The issued token contains approved scopes, which the Resource Server enforces.
  }
\end{frame}

% JWT Bearer configuration in ASP.NET Core
\QuestionSlide[\CategoryBadge[AuthColor!20]{Security}]{** How do you configure JWT Bearer Authentication in ASP\.NET Core?}

\begin{frame}[fragile]
  \frametitle{%
    \begin{tikzpicture}[remember picture, overlay]
      \node[anchor=north east, xshift=-0.4cm, yshift=-0.4cm, text=black] at (current page.north east) {
        \CategoryBadge[OOPColor!20]{C\#}
        \CategoryBadge[AuthColor!20]{Security}
      };
    \end{tikzpicture}
    Answer \theqcounter: How do you configure JWT Bearer Authentication in ASP\.NET Core?%
  }

  {\footnotesize
  Use the `JwtBearer` middleware in `Startup`/`Program.cs`, specifying `Authority` and `Audience`:
  }

  \begin{minted}[fontsize=\scriptsize]{csharp}
services.AddAuthentication(JwtBearerDefaults.AuthenticationScheme)
  .AddJwtBearer(options => {
    options.Authority = "https://idp.example.com";
    options.Audience = "api1";
    options.RequireHttpsMetadata = true;
  });
  \end{minted}
\end{frame}

% OpenID Connect – what is it?
\QuestionSlide[\CategoryBadge[AuthColor!20]{Security}]{** What is OpenID Connect?}

\begin{frame}[fragile]
  \frametitle{%
    \begin{tikzpicture}[remember picture, overlay]
      \node[anchor=north east, xshift=-0.4cm, yshift=-0.4cm, text=black] at (current page.north east) {
        \CategoryBadge[OOPColor!20]{C\#}
        \CategoryBadge[AuthColor!20]{Security}
      };
    \end{tikzpicture}
    Answer \theqcounter: What is OpenID Connect?%
  }

  {\footnotesize
  OpenID Connect is an authentication layer built on OAuth 2.0. It adds an ID Token (JWT) containing user identity claims, plus discovery and UserInfo endpoints.
  }
\end{frame}

% ID Token
\QuestionSlide[\CategoryBadge[AuthColor!20]{Security}]{** What is an ID Token?}

\begin{frame}[fragile]
  \frametitle{%
    \begin{tikzpicture}[remember picture, overlay]
      \node[anchor=north east, xshift=-0.4cm, yshift=-0.4cm, text=black] at (current page.north east) {
        \CategoryBadge[OOPColor!20]{C\#}
        \CategoryBadge[AuthColor!20]{Security}
      };
    \end{tikzpicture}
    Answer \theqcounter: What is an ID Token?%
  }

  {\footnotesize
  An ID Token is a JWT issued by the OpenID Provider during the Authorization Code Flow, containing user identity claims such as `sub`, `email`, and `name`.
  }
\end{frame}

% OIDC configuration in ASP.NET Core
\QuestionSlide[\CategoryBadge[AuthColor!20]{Security}]{** How do you implement OpenID Connect in ASP\.NET Core?}

\begin{frame}[fragile]
  \frametitle{%
    \begin{tikzpicture}[remember picture, overlay]
      \node[anchor=north east, xshift=-0.4cm, yshift=-0.4cm, text=black] at (current page.north east) {
        \CategoryBadge[OOPColor!20]{C\#}
        \CategoryBadge[AuthColor!20]{Security}
      };
    \end{tikzpicture}
    Answer \theqcounter: How do you implement OpenID Connect in ASP\.NET Core?%
  }

  {\footnotesize
  Configure the OIDC scheme in `Program.cs`:
  }

  \begin{minted}[fontsize=\scriptsize]{csharp}
services.AddAuthentication(options => {
  options.DefaultScheme = "Cookies";
  options.DefaultChallengeScheme = "oidc";
})
.AddCookie("Cookies")
.AddOpenIdConnect("oidc", options => {
  options.Authority = "https://idp.example.com";
  options.ClientId = "mvc_client";
  options.ClientSecret = "secret";
  options.ResponseType = "code";
  options.Scope.Add("openid");
  options.Scope.Add("profile");
  options.SaveTokens = true;
});
  \end{minted}
\end{frame}

% OAuth 2.0 vs OpenID Connect
\QuestionSlide[\CategoryBadge[AuthColor!20]{Security}]{** What is the difference between OAuth 2.0 and OpenID Connect?}

\begin{frame}[fragile]
  \frametitle{%
    \begin{tikzpicture}[remember picture, overlay]
      \node[anchor=north east, xshift=-0.4cm, yshift=-0.4cm, text=black] at (current page.north east) {
        \CategoryBadge[OOPColor!20]{C\#}
        \CategoryBadge[AuthColor!20]{Security}
      };
    \end{tikzpicture}
    Answer \theqcounter: What is the difference between OAuth 2.0 and OpenID Connect?%
  }

  {\footnotesize
  OAuth 2.0 is a protocol for delegating access (authorization).\\
  OpenID Connect extends OAuth 2.0 by adding authentication, issuing an ID Token that verifies the user’s identity.
  }
\end{frame}

% 1 ── ASP.NET Core Identity
\QuestionSlide[\CategoryBadge[AuthColor!20]{Security}]{** What is ASP\.NET Core Identity?}

\begin{frame}[fragile]
  \frametitle{%
    \begin{tikzpicture}[remember picture, overlay]
      \node[anchor=north east, xshift=-0.4cm, yshift=-0.4cm, text=black] at (current page.north east) {
        \CategoryBadge[OOPColor!20]{C\#}
        \CategoryBadge[AuthColor!20]{Security}
      };
    \end{tikzpicture}
    Answer \theqcounter: What is ASP\.NET Core Identity?%
  }

  {\footnotesize
  ASP\.NET Core Identity is a membership system that adds login, logout, user management, roles, claims, and storing user data (via EF Core by default) to your application.
  }

  \begin{minted}{csharp}
// In Program.cs
services.AddDefaultIdentity<IdentityUser>()
    .AddEntityFrameworkStores<ApplicationDbContext>();
  \end{minted}
\end{frame}

% 2 ── Cookie Authentication
\QuestionSlide[\CategoryBadge[AuthColor!20]{Security}]{** What is Cookie Authentication in ASP\.NET Core?}

\begin{frame}[fragile]
  \frametitle{%
    \begin{tikzpicture}[remember picture, overlay]
      \node[anchor=north east, xshift=-0.4cm, yshift=-0.4cm, text=black] at (current page.north east) {
        \CategoryBadge[OOPColor!20]{C\#}
        \CategoryBadge[AuthColor!20]{Security}
      };
    \end{tikzpicture}
    Answer \theqcounter: What is Cookie Authentication in ASP\.NET Core?%
  }

  {\footnotesize
  Cookie Authentication issues an encrypted cookie after login, which is sent on each request to identify the user session. It’s suitable for web apps that maintain user sessions.
  }

  \begin{minted}{csharp}
services.AddAuthentication(CookieAuthenticationDefaults.AuthenticationScheme)
    .AddCookie(options => {
      options.LoginPath = "/Account/Login";
      options.ExpireTimeSpan = TimeSpan.FromHours(1);
    });
  \end{minted}
\end{frame}

% 3 ── Claims-based Authorization
\QuestionSlide[\CategoryBadge[AuthColor!20]{Security}]{** What is Claims-based Authorization?}

\begin{frame}[fragile]
  \frametitle{%
    \begin{tikzpicture}[remember picture, overlay]
      \node[anchor=north east, xshift=-0.4cm, yshift=-0.4cm, text=black] at (current page.north east) {
        \CategoryBadge[OOPColor!20]{C\#}
        \CategoryBadge[AuthColor!20]{Security}
      };
    \end{tikzpicture}
    Answer \theqcounter: What is Claims-based Authorization?%
  }

  {\footnotesize
  Claims-based Authorization grants or denies access based on claims in the user’s identity (e.g., roles, permissions). Policies evaluate claims via requirements and handlers.
  }

  \begin{minted}{csharp}
// Define policy
services.AddAuthorization(options => {
  options.AddPolicy("CanEdit", policy =>
    policy.RequireClaim("Permission", "Edit"));
});

// Enforce in controller
[Authorize(Policy = "CanEdit")]
public IActionResult Edit() => View();
  \end{minted}
\end{frame}

% 4 ── Roles vs Policies
\QuestionSlide[\CategoryBadge[AuthColor!20]{Security}]{** What is the difference between Roles and Policies?}

\begin{frame}[fragile]
  \frametitle{%
    \begin{tikzpicture}[remember picture, overlay]
      \node[anchor=north east, xshift=-0.4cm, yshift=-0.4cm, text=black] at (current page.north east) {
        \CategoryBadge[OOPColor!20]{C\#}
        \CategoryBadge[AuthColor!20]{Security}
      };
    \end{tikzpicture}
    Answer \theqcounter: What is the difference between Roles and Policies?%
  }

  {\footnotesize
  Roles assign users to named groups (e.g., “Admin”). Policies are more flexible rules combining multiple claims, roles, or custom requirements.
  }

  \begin{minted}{csharp}
// Role-based
[Authorize(Roles = "Admin")]

// Policy-based
[Authorize(Policy = "CanEdit")]
  \end{minted}
\end{frame}

% 5 ── Refresh Tokens in .NET
\QuestionSlide[\CategoryBadge[AuthColor!20]{Security}]{** What is a Refresh Token and how is it used?}

\begin{frame}[fragile]
  \frametitle{%
    \begin{tikzpicture}[remember picture, overlay]
      \node[anchor=north east, xshift=-0.4cm, yshift=-0.4cm, text=black] at (current page.north east) {
        \CategoryBadge[OOPColor!20]{C\#}
        \CategoryBadge[AuthColor!20]{Security}
      };
    \end{tikzpicture}
    Answer \theqcounter: What is a Refresh Token and how is it used?%
  }

  {\footnotesize
  A Refresh Token is a long-lived token used to obtain new access tokens when they expire. It’s stored securely and sent to the token endpoint in exchange for fresh tokens.
  }

  \begin{minted}{csharp}
// Exchange refresh token
var tokenResponse = await httpClient.RequestRefreshTokenAsync(new RefreshTokenRequest {
  Address = "https://idp.example.com/connect/token",
  ClientId = "client",
  ClientSecret = "secret",
  RefreshToken = storedRefreshToken
});
  \end{minted}
\end{frame}

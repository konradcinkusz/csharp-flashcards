% 1 --- Handling poison messages in Azure Service Bus
\QuestionSlide[\CategoryBadge[AuthColor!20]{Cloud}]{*** How do you handle poison messages in Azure Service Bus queues?}
\begin{frame}[fragile]
  \frametitle{
    \begin{tikzpicture}[remember picture,overlay]
      \node[anchor=north east,xshift=-0.4cm,yshift=-0.4cm] at (current page.north east) {
        \CategoryBadge[AuthColor!20]{Cloud}
      };
    \end{tikzpicture}

    Answer \theqcounter: *** How do you handle poison messages in Azure Service Bus queues?%
  }

  {\footnotesize
  \begin{itemize}
    \item \textbf{Detect} when a message exceeds \texttt{maxDeliveryCount}; Service Bus moves it to the dead-letter queue.
    \item \textbf{Process} the DLQ with a separate worker that logs and decides whether to fix or discard the payload.
    \item \textbf{Mitigate} with exponential back-off retries and idempotent handlers to avoid repeats.
    \item \textbf{Prevent} by validating input early and monitoring \texttt{DeadLetterMessageCount}.
  \end{itemize}
  }
\end{frame}

% 2 --- Split vs. single queries in EF Core 8
\QuestionSlide[\CategoryBadge[AuthColor!20]{Cloud}]{** When should you prefer split queries over single queries in EF Core 8?}
\begin{frame}[fragile]
  \frametitle{
    \begin{tikzpicture}[remember picture,overlay]
      \node[anchor=north east,xshift=-0.4cm,yshift=-0.4cm] at (current page.north east) {
        \CategoryBadge[AuthColor!20]{Cloud}
      };
    \end{tikzpicture}

    Answer \theqcounter: ** When should you prefer split queries over single queries in EF Core 8?%
  }

  {\footnotesize
  \begin{itemize}
    \item \textbf{Single query} joins all includes in one SQL statement; efficient but can explode row counts.
    \item \textbf{Split query} runs separate SQL per include via \texttt{AsSplitQuery}; avoids duplication at the cost of extra round trips.
    \item Choose split queries for large graphs or collection includes that would return many duplicate rows.
    \item Configure globally with \texttt{UseQuerySplittingBehavior} or per query using \texttt{AsSingleQuery}/\texttt{AsSplitQuery}.
  \end{itemize}
  }

  \begin{minted}[fontsize=\scriptsize]{csharp}
var blogs = context.Blogs
    .Include(b => b.Posts)
    .AsSplitQuery()
    .ToList();
  \end{minted}
\end{frame}

% 3 --- AKS rolling updates on probe failures
\QuestionSlide[\CategoryBadge[AuthColor!20]{Cloud}]{** What happens during AKS rolling updates when health probes fail?}
\begin{frame}[fragile]
  \frametitle{
    \begin{tikzpicture}[remember picture,overlay]
      \node[anchor=north east,xshift=-0.4cm,yshift=-0.4cm] at (current page.north east) {
        \CategoryBadge[AuthColor!20]{Cloud}
      };
    \end{tikzpicture}

    Answer \theqcounter: ** What happens during AKS rolling updates when health probes fail?%
  }

  {\footnotesize
  \begin{itemize}
    \item Kubernetes creates a new pod and waits for readiness before routing traffic.
    \item Failed readiness or liveness probes keep the pod \textbf{NotReady}, pausing the rollout.
    \item If failures exceed \texttt{progressDeadlineSeconds}, the deployment is marked failed and can auto-roll back.
    \item Monitor with \texttt{kubectl rollout status} and undo with \texttt{kubectl rollout undo}.
  \end{itemize}
  }

  \begin{minted}[fontsize=\scriptsize]{bash}
kubectl rollout status deployment/myapp
  \end{minted}
\end{frame}

% 4 --- ROW_NUMBER() optimizations
\QuestionSlide[\CategoryBadge[AuthColor!20]{Cloud}]{** How can you optimize queries that use ROW\_NUMBER() in SQL Server?}
\begin{frame}[fragile]
  \frametitle{
    \begin{tikzpicture}[remember picture,overlay]
      \node[anchor=north east,xshift=-0.4cm,yshift=-0.4cm] at (current page.north east) {
        \CategoryBadge[AuthColor!20]{Cloud}
      };
    \end{tikzpicture}

    Answer \theqcounter: ** How can you optimize queries that use ROW\_NUMBER() in SQL Server?%
  }

  {\footnotesize
  \begin{itemize}
    \item Index the \texttt{PARTITION BY} and \texttt{ORDER BY} columns to avoid expensive sorts.
    \item Filter in a CTE or subquery so the engine streams rows instead of materializing the whole set.
    \item Use \texttt{TOP (1) WITH TIES} when only the first row per group is required.
  \end{itemize}
  }

  \begin{minted}[fontsize=\scriptsize]{sql}
WITH ranked AS (
  SELECT *, ROW_NUMBER() OVER (
      PARTITION BY CustomerId ORDER BY Created DESC) AS rn
  FROM Orders WITH (INDEX(IX_Orders_CustomerId_Created))
)
SELECT * FROM ranked WHERE rn = 1;
  \end{minted}
\end{frame}

% 5 --- dotnet-trace & PerfView for LOH
\QuestionSlide[\CategoryBadge[AuthColor!20]{Cloud}]{** How do you investigate Large Object Heap issues with dotnet-trace and PerfView?}
\begin{frame}[fragile]
  \frametitle{
    \begin{tikzpicture}[remember picture,overlay]
      \node[anchor=north east,xshift=-0.4cm,yshift=-0.4cm] at (current page.north east) {
        \CategoryBadge[AuthColor!20]{Cloud}
      };
    \end{tikzpicture}

    Answer \theqcounter: ** How do you investigate Large Object Heap issues with dotnet-trace and PerfView?%
  }

  {\footnotesize
  \begin{itemize}
    \item Collect GC events with \texttt{dotnet-trace collect --process-id <pid> --profile gc}.
    \item Open the resulting \texttt{.nettrace} file in PerfView to inspect \textbf{GC Heap Alloc Stacks}.
    \item Focus on objects \(\geq 85\,KB\) to find excessive LOH allocations and leaks.
  \end{itemize}
  }

  \begin{minted}[fontsize=\scriptsize]{bash}
dotnet-trace collect --process-id 1234 --profile gc
perfview trace.nettrace
  \end{minted}
\end{frame}

% 6 --- Server GC mode
\QuestionSlide[\CategoryBadge[AuthColor!20]{Cloud}]{** When should you enable Server GC mode and how is it configured?}
\begin{frame}[fragile]
  \frametitle{
    \begin{tikzpicture}[remember picture,overlay]
      \node[anchor=north east,xshift=-0.4cm,yshift=-0.4cm] at (current page.north east) {
        \CategoryBadge[AuthColor!20]{Cloud}
      };
    \end{tikzpicture}

    Answer \theqcounter: ** When should you enable Server GC mode and how is it configured?%
  }

  {\footnotesize
  \begin{itemize}
    \item Server GC creates a separate heap per core and multiple GC threads for high-throughput scenarios.
    \item Enable it in background services or web servers; avoid on small or GUI apps due to higher memory use.
    \item Configure via \texttt{<ServerGarbageCollection>true</ServerGarbageCollection>} or environment variable.
  \end{itemize}
  }

  \begin{minted}[fontsize=\scriptsize]{xml}
<PropertyGroup>
  <ServerGarbageCollection>true</ServerGarbageCollection>
</PropertyGroup>
  \end{minted}
\end{frame}

% 7 --- OpenTelemetry + Tempo tracing
\QuestionSlide[\CategoryBadge[AuthColor!20]{Cloud}]{** How do you send OpenTelemetry traces to Grafana Tempo?}
\begin{frame}[fragile]
  \frametitle{
    \begin{tikzpicture}[remember picture,overlay]
      \node[anchor=north east,xshift=-0.4cm,yshift=-0.4cm] at (current page.north east) {
        \CategoryBadge[AuthColor!20]{Cloud}
      };
    \end{tikzpicture}

    Answer \theqcounter: ** How do you send OpenTelemetry traces to Grafana Tempo?%
  }

  {\footnotesize
  \begin{itemize}
    \item Tempo accepts OTLP; point the exporter to the Tempo endpoint.
    \item Use \texttt{AddOpenTelemetry} with \texttt{AddOtlpExporter} or set \texttt{OTEL\_EXPORTER\_OTLP\_ENDPOINT}.
    \item Ensure trace context propagation so spans from different services join in Tempo.
    \item Visualize traces by adding Tempo as a data source in Grafana.
  \end{itemize}
  }

  \begin{minted}[fontsize=\scriptsize]{csharp}
builder.Services.AddOpenTelemetry()
    .WithTracing(b => b
        .AddAspNetCoreInstrumentation()
        .AddOtlpExporter(o => o.Endpoint = new("http://tempo:4317")));
  \end{minted}
\end{frame}

% 8 --- Security headers for Angular micro-frontends
\QuestionSlide[\CategoryBadge[AuthColor!20]{Cloud}]{** Which HTTP security headers help protect Angular micro-frontends?}
\begin{frame}[fragile]
  \frametitle{
    \begin{tikzpicture}[remember picture,overlay]
      \node[anchor=north east,xshift=-0.4cm,yshift=-0.4cm] at (current page.north east) {
        \CategoryBadge[AuthColor!20]{Cloud}
      };
    \end{tikzpicture}

    Answer \theqcounter: ** Which HTTP security headers help protect Angular micro-frontends?%
  }

  {\footnotesize
  \begin{itemize}
    \item \textbf{Content-Security-Policy} limits scripts and styles to trusted origins and isolates micro-frontends.
    \item \textbf{X-Frame-Options} or \texttt{frame-ancestors} prevents clickjacking.
    \item \textbf{Strict-Transport-Security} enforces HTTPS for all requests.
    \item \textbf{X-Content-Type-Options: nosniff} and \textbf{Referrer-Policy: same-origin} reduce attack surface.
  \end{itemize}
  }

  \begin{minted}[fontsize=\scriptsize]{nginx}
add_header Content-Security-Policy "default-src 'self' https://cdn.example";
add_header X-Content-Type-Options "nosniff";
  \end{minted}
\end{frame}

% 9 --- Handling risky hotfix requests
\QuestionSlide[\CategoryBadge[AuthColor!20]{Cloud}]{** How do you handle urgent hotfix requests that carry significant risk?}
\begin{frame}[fragile]
  \frametitle{
    \begin{tikzpicture}[remember picture,overlay]
      \node[anchor=north east,xshift=-0.4cm,yshift=-0.4cm] at (current page.north east) {
        \CategoryBadge[AuthColor!20]{Cloud}
      };
    \end{tikzpicture}

    Answer \theqcounter: ** How do you handle urgent hotfix requests that carry significant risk?%
  }

  {\footnotesize
  \begin{itemize}
    \item Clarify business impact and verify the issue to avoid unnecessary changes.
    \item Scope the fix in a dedicated branch and guard it with a feature flag or configuration switch.
    \item Run targeted tests and seek peer review before deploying.
    \item Communicate a rollback plan and deployment window to stakeholders.
  \end{itemize}
  }
\end{frame}

% 10 --- Reducing lead time
\QuestionSlide[\CategoryBadge[AuthColor!20]{Cloud}]{** What practices help reduce lead time from commit to production?}
\begin{frame}[fragile]
  \frametitle{
    \begin{tikzpicture}[remember picture,overlay]
      \node[anchor=north east,xshift=-0.4cm,yshift=-0.4cm] at (current page.north east) {
        \CategoryBadge[AuthColor!20]{Cloud}
      };
    \end{tikzpicture}

    Answer \theqcounter: ** What practices help reduce lead time from commit to production?%
  }

  {\footnotesize
  \begin{itemize}
    \item Automate build, test, and deployment stages with a fast CI/CD pipeline.
    \item Use trunk-based development and small pull requests to minimize integration delays.
    \item Invest in reliable automated tests and static analysis to eliminate manual gates.
    \item Deploy behind feature flags and use progressive delivery to decouple release from deploy.
  \end{itemize}
  }
\end{frame}

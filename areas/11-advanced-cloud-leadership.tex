% 1 --- Handling poison messages in Azure Service Bus
\QuestionSlide[\CategoryBadge[AuthColor!20]{Cloud}]{*** How do you handle poison messages in Azure Service Bus queues?}
\begin{frame}[fragile]
  \frametitle{
    \begin{tikzpicture}[remember picture,overlay]
      \node[anchor=north east,xshift=-0.4cm,yshift=-0.4cm] at (current page.north east) {
        \CategoryBadge[AuthColor!20]{Cloud}
      };
    \end{tikzpicture}

    Answer \theqcounter: *** How do you handle poison messages in Azure Service Bus queues?%
  }

  {\footnotesize
  \begin{itemize}
    \item \textbf{Detect} when a message exceeds \texttt{maxDeliveryCount}; Service Bus moves it to the dead-letter queue.
    \item \textbf{Process} the DLQ with a separate worker that logs and decides whether to fix or discard the payload.
    \item \textbf{Mitigate} with exponential back-off retries and idempotent handlers to avoid repeats.
    \item \textbf{Prevent} by validating input early and monitoring \texttt{DeadLetterMessageCount}.
  \end{itemize}
  }
\end{frame}

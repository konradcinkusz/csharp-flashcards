% 1 ── What is LINQ?
\QuestionSlide[\CategoryBadge[LinqColor!20]{LINQ}]{** What is LINQ?}
\begin{frame}[fragile]
  \frametitle{%
    \begin{tikzpicture}[remember picture, overlay]
      \node[anchor=north east, xshift=-0.4cm, yshift=-0.4cm, text=black] at (current page.north east) {
        \CategoryBadge[LinqColor!20]{LINQ}
      };
    \end{tikzpicture}
    Answer \theqcounter: What is LINQ?%
  }
  {\footnotesize
  Language Integrated Query enables querying collections with SQL-like syntax or fluent methods (\texttt{Where}, \texttt{Select}, etc.).
  }
\end{frame}

% 2 ── .Select() in LINQ
\QuestionSlide[\CategoryBadge[LinqColor!20]{LINQ}]{* What does \texttt{.Select()} do in LINQ?}
\begin{frame}[fragile]
  \frametitle{%
    \begin{tikzpicture}[remember picture, overlay]
      \node[anchor=north east, xshift=-0.4cm, yshift=-0.4cm, text=black] at (current page.north east) {
        \CategoryBadge[LinqColor!20]{LINQ}
      };
    \end{tikzpicture}
    Answer \theqcounter: What does \texttt{.Select()} do in LINQ?%
  }
  {\footnotesize
  \texttt{.Select()} projects each element of a sequence into a new form. It returns a collection where each element corresponds 1-to-1 with the input sequence.
  }
  \begin{minted}{csharp}
  // Get customer names from WorldWideImporters
  var names = context.Customers
                     .Select(c => c.CustomerName)
                     .ToList();
  \end{minted}
\end{frame}

% 3 ── .SelectMany() in LINQ
\QuestionSlide[\CategoryBadge[LinqColor!20]{LINQ}]{** What does \texttt{.SelectMany()} do in LINQ?}
\begin{frame}[fragile]
  \frametitle{%
    \begin{tikzpicture}[remember picture, overlay]
      \node[anchor=north east, xshift=-0.4cm, yshift=-0.4cm, text=black] at (current page.north east) {
        \CategoryBadge[LinqColor!20]{LINQ}
      };
    \end{tikzpicture}
    Answer \theqcounter: What does \texttt{.SelectMany()} do in LINQ?%
  }
  {\footnotesize
  \texttt{.SelectMany()} flattens nested collections. It projects each element to a collection, then merges all collections into a single flat result set.
  }
  \begin{minted}{csharp}
  // Get all InvoiceLines from all Invoices
  var allLines = context.Invoices
                        .SelectMany(i => i.InvoiceLines)
                        .ToList();
  \end{minted}
\end{frame}

% 4 ── When to use .SelectMany()
\QuestionSlide[\CategoryBadge[LinqColor!20]{LINQ}]{** When would you use \texttt{.SelectMany()} instead of \texttt{.Select()}?}
\begin{frame}[fragile]
  \frametitle{%
    \begin{tikzpicture}[remember picture, overlay]
      \node[anchor=north east, xshift=-0.4cm, yshift=-0.4cm, text=black] at (current page.north east) {
        \CategoryBadge[LinqColor!20]{LINQ}
      };
    \end{tikzpicture}
    Answer \theqcounter: When would you use \texttt{.SelectMany()} instead of \texttt{.Select()}?%
  }
  {\footnotesize
  Use \texttt{.SelectMany()} when projecting nested collections and you want to work with their individual elements directly in a flat structure.
  }
  \begin{minted}{csharp}
  // .Select() keeps nesting
  var nested = context.Customers
                      .Select(c => c.Orders)
                      .ToList(); // List<ICollection<Order>>

  // .SelectMany() flattens to List<Order>
  var flat = context.Customers
                    .SelectMany(c => c.Orders)
                    .ToList();
  \end{minted}
  {\footnotesize
  \texttt{.Select()} → nested structure  
  \texttt{.SelectMany()} → flat structure
  }
\end{frame}

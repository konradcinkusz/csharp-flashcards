% Microservices section

% 1 – Bulkhead Pattern
\QuestionSlide[\CategoryBadge[MicroColor!20]{Microservices}]{** What is the Bulkhead Pattern?}

\begin{frame}[fragile]
  \frametitle{%
    \begin{tikzpicture}[remember picture, overlay]
      \node[anchor=north east, xshift=-0.4cm, yshift=-0.4cm, text=black] at (current page.north east) {
        \CategoryBadge[MicroColor!20]{Microservices}
      };
    \end{tikzpicture}
    Answer \theqcounter: What is the Bulkhead Pattern?%
  }

  {\footnotesize
  The Bulkhead Pattern isolates critical resources (threads, connections) into independent pools so that a failure in one area does not bring down the entire system.
  }

  \begin{minted}[fontsize=\scriptsize]{csharp}
var bulkhead = Policy.BulkheadAsync<HttpResponseMessage>(
    maxParallelization: 50,
    maxQueuingActions: 100);
await bulkhead.ExecuteAsync(() => httpClient.GetAsync(url));
  \end{minted}
\end{frame}

% 2 – Sidecar Pattern
\QuestionSlide[\CategoryBadge[MicroColor!20]{Microservices}]{* What is the Sidecar Pattern?}

\begin{frame}[fragile]
  \frametitle{%
    \begin{tikzpicture}[remember picture, overlay]
      \node[anchor=north east, xshift=-0.4cm, yshift=-0.4cm, text=black] at (current page.north east) {
        \CategoryBadge[MicroColor!20]{Microservices}
      };
    \end{tikzpicture}
    Answer \theqcounter: What is the Sidecar Pattern?%
  }

  {\footnotesize
  The Sidecar Pattern deploys a helper service alongside the main service within the same host or pod. The sidecar handles cross-cutting concerns like logging or proxying without modifying the primary service.
  }

  \begin{minted}[fontsize=\scriptsize]{yaml}
apiVersion: v1
kind: Pod
metadata:
  name: app
spec:
  containers:
  - name: web
    image: myapi:1.0
  - name: logger
    image: fluentd:latest
  \end{minted}
\end{frame}

% 3 – Service Mesh
\QuestionSlide[\CategoryBadge[MicroColor!20]{Microservices}]{** What is a Service Mesh?}

\begin{frame}[fragile]
  \frametitle{%
    \begin{tikzpicture}[remember picture, overlay]
      \node[anchor=north east, xshift=-0.4cm, yshift=-0.4cm, text=black] at (current page.north east) {
        \CategoryBadge[MicroColor!20]{Microservices}
      };
    \end{tikzpicture}
    Answer \theqcounter: What is a Service Mesh?%
  }

  {\footnotesize
  A Service Mesh is an infrastructure layer that manages service-to-service communication via sidecar proxies, providing features like traffic routing, retries, and observability without changing application code. Popular meshes include Istio and Linkerd.
  }

  \begin{minted}[fontsize=\scriptsize]{bash}
# Enable Istio sidecar injection
kubectl label namespace default istio-injection=enabled
  \end{minted}
\end{frame}

% 4 – Strangler Fig Pattern
\QuestionSlide[\CategoryBadge[MicroColor!20]{Microservices}]{* What is the Strangler Fig Pattern?}

\begin{frame}[fragile]
  \frametitle{%
    \begin{tikzpicture}[remember picture, overlay]
      \node[anchor=north east, xshift=-0.4cm, yshift=-0.4cm, text=black] at (current page.north east) {
        \CategoryBadge[MicroColor!20]{Microservices}
      };
    \end{tikzpicture}
    Answer \theqcounter: What is the Strangler Fig Pattern?%
  }

  {\footnotesize
  The Strangler Fig Pattern incrementally replaces pieces of a monolith with microservices. A facade routes requests between legacy and new components until the monolith can be retired.
  }

  \begin{minted}[fontsize=\scriptsize]{csharp}
// ASP.NET Core facade routing
app.MapWhen(ctx => ctx.Request.Path.StartsWithSegments("/legacy"),
    legacy => legacy.RunProxy("http://monolith"));
app.Map("/orders", m => m.RunProxy("http://orders-service"));
  \end{minted}
\end{frame}

% 5 – Event Sourcing
\QuestionSlide[\CategoryBadge[MicroColor!20]{Microservices}]{** What is Event Sourcing?}

\begin{frame}[fragile]
  \frametitle{%
    \begin{tikzpicture}[remember picture, overlay]
      \node[anchor=north east, xshift=-0.4cm, yshift=-0.4cm, text=black] at (current page.north east) {
        \CategoryBadge[MicroColor!20]{Microservices}
      };
    \end{tikzpicture}
    Answer \theqcounter: What is Event Sourcing?%
  }

  {\footnotesize
  Event Sourcing persists state as a sequence of immutable events. The current state is rebuilt by replaying events, enabling full audit history and temporal queries.
  }

  \begin{minted}[fontsize=\scriptsize]{csharp}
public record FundsDeposited(Guid AccountId, decimal Amount);

public class Account {
    private decimal _balance;
    public void Apply(FundsDeposited e) => _balance += e.Amount;
}
  \end{minted}
\end{frame}
